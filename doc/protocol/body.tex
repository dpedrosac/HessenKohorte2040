\setstretch{1.5}
\chapter{Protocol}
\section{Introduction}
\subsection{Background}
\acl{iPD}\acused{iPD} represents a chronic neurodegenerative disease manifested by both motor and non-motor symptoms. The physical impairments in \ac{iPD} have a significant psychosocial impact and lead to considerable losses in patients' quality of life \ac(QoL) and a high burden on (informal) caregivers (source: ??). Several quality of life assessment tools have been developed so far, some of which are specific to \ac{iPS} (source: 10.3390/jpm12050804). However, none of the models take into account positive aspects of well-being or a person's personal attitude (e.g. optimism) but also aspects such as stress factors, like social support, level of integration, to name a few. In the present project, an investigation of the quality of life of \ac{iPD}patients over time will be carried out. For this purpose, quality of life will be assessed longitudinally using established and validated \ac{HRQOL}-questionnaires. In addition, holistic observations of quality of life should lead to meaningful statements. The instrument intended for this purpose is the so-called \ac{CHAPO}-model, an approach originally developed for the assessment of the quality of life of very old people (\url{https://ceres.uni-koeln.de/forschung/nrw80}). By adapting it to aspects of \ac{iPD}-patients, the so-called \acs{CHAPO}-PD-model (source: 10.3233/JPD-202391) will be applied in this cohort study. The aim of this project is thus to record quality of life with the aim of recording \acs{iPD} patients in a standardised way over the entire course of the disease. We would like to link the data obtained in this way with an annual follow-up that includes a cranial \ac{MRI} and the biomedical markers obtained from stool, urine, saliva, hair and blood samples. In this way, imaging or biomedical markers with predictive value for quality of life change could be identified. In addition, this longitudinal study will include a follow-up assessment of the support services needed to better understand the needs of family members of \ac{iPD}patients. This should enable the development of a needs-based support service according to the different phases of the disease. Stress experience, changes in sleep patterns and quality of life losses over the observation period will also be included in the analysis to identify a surrogate for adequate support.

\subsection{Geographic context}
\ac{iPD} is one of the most common neurological diseases. Estimates put the incidence of the disease in Germany at \SI{84.1}{} per \SI{100000}{} people and per year and assume a number of approx. \SI{400000}{} people (source: 111/ane.12694). In order to understand the special features of the \UKGM as far as the care of \ac{iPD} patients is concerned, it is first necessary to know the location of Marburg. There live approx. \SI{77000}{} live in the city of Marburg and it is located in the countryside in the western centre of Germany [18]. It is a university town and a district town in the federal state of Hesse. Due to its location at about \SI{80}{\km} direct distance between the metropolitan areas of Frankfurt am Main and Kassel, the role of the \UKGM must be understood as the predominant centre for medical care in the district. About \SI{1500}{} people with a \ac{iPD} are treated at the \UKGM every year. In order to ensure that patients have access to care in the district of Marburg, the \ac{PANAMA} was founded in 2016 by the Clinic for Neurology of the \UKGM. In this care network, different actors work together to facilitate the integration of care services and improve outcomes for patients. At the same time, it is a tertiary centre that combines established treatment services for each stage of the disease with university medicine and offers a series of studies that can provide innovative forms of therapy. This makes it possible to offer modern and tailor-made therapy to people from beyond the region.

In order to represent the diversity of the real population of \ac{iPD}-patients of \UKGM and to ensure a balanced study cohort, as many of the treated patients as possible should be given the opportunity to participate in the study. Accordingly, the management of the study imposes on itself to adopt the recruitment strategies that have been successfully tested in previous clinical studies and to adapt them to the requirements of the long-term cohort study. To this end, on the one hand, patients are directly offered participation in the study during their appointments in the outpatient clinic of the hospital or during their inpatient stay in the Clinic for Neurology of the \UKGM. Secondly, members of the \ac{PANAMA} network will be made aware of the study with the aim of arousing the interest of potential participants. Finally, detailed information will be made available on the media website of \UKGM to ensure sufficient information (\url{https://www.uni-marburg.de}).

\newpage

%%%%%%%%%%%%%%%%%%%%%%%%%%%%%%%%%%%%%%%%%%%%%%%%%%%%%%%%%%%%%%%%%%%%%%%
%%%% 										START OF SYNOPSIS												%%%%
%%%%%%%%%%%%%%%%%%%%%%%%%%%%%%%%%%%%%%%%%%%%%%%%%%%%%%%%%%%%%%%%%%%%%%%

\setstretch{1}
\section{Protocol synopsis}
\begin{tabularx}{1\textwidth}{m{3.5cm} | X}
\toprule
%\multicolumn{2}{c}{}\\
\multicolumn{2}{p{\dimexpr\linewidth-2\tabcolsep-2\arrayrulewidth}|}
{\textbf{
 Longitudinal digital observation of the holistic quality of
 the life of patients with \ac{iPD} and their caregivers:
 a prospective observational cohort study
}}
\\ \toprule

\textbf{Study objectives} & 
This study aims at observing quality of life of \SI{1000} patients suffering from \ac{iPD} and their relatives over the course of 20 years and relating this to a objectifiable changes in the metabolism but also to structural imaging changes during this time.
\\ \midrule

\textbf{Study design} &
Prospective single-center cohort study
\\ \midrule

\textbf{Planned Number of Subjects} &
\SI{1000}{}
\\ \midrule

\textbf{Primary Endpoint} &
Quality of life of the index patient after an observation of up to 20 years
\\ \midrule

\textbf{Secondary Endpoints} & 
\tabitem{Index patients' relatives quality of life after an observation of up to 20 years} \\
& \tabitem{Changes in motor symptoms of the index patient after up to 20 years} \\
& \tabitem{Development of non-motor symptoms over up to 20 years} \\
& \tabitem{Changes of the functional imaging over the observational period} 
\\ \midrule

\textbf{Enrollment of participants} & Patients suffering from \ac{iPD} may be enrolled together with their relatives at any point in time.
\\ \midrule

\textbf{Study visits schedule} & 
\tabitem{Screening} \\
& \tabitem{Baseline Visit}\\
& \tabitem{Yearly follow-up}\\
& \tabitem{\ldots}\\
& \tabitem{Visit at year 2042 (\textit{End-of-Study}-visit)} 
\\ \midrule 

\textbf{Study Duration} &
The study will be considered complete after all subjects complete their visit in the year 2042. Hence, the total study duration is estimated to be at most 20 years.
\\ \midrule

\textbf{Inclusion criteria \ac{iPD}-patients} &
\tabitem{Patients suffering from a clinical diagnosis of idiopathic Parkinson's syndrome according to the recent clinical diagnostic criteria \citep{postuma2015mds}} \\
& \tabitem{\ac{iPD}-stages of \RNum{1} -- \RNum{4} according to the Hoehn \& Yahr
  scale (in the OFF state, i.e., without medication) \citep{hoehn1967parkinsonism}} \\
& \tabitem{Patients aged between between 30 and 100 years} \\
& \tabitem{Patients with the ability to provide informed consent. In
  cases where participants lose their capacity to consent at follow-up
  visits (e.g., due to dementia, etc.), this participant will only be
  allowed to continue if a legal representative (proxy, guardian)
  provides informed consent to further participation on behalf of the
  participant. In this case, the legal representatives will be
  provided with a separate consent form.} \\
& \tabitem{Patients with a good knowledge of German}
\\ \midrule

\textbf{Exclusion criteria \ac{PD}-patients} &
\tabitem{Patients suffering from a clinical diagnosis of atypical 
  Parkinson's syndrome in a first instance. Patients enrolled who
  were later characterized as atypical Parkinson syndroms will not be
  excluded.}\\
& \tabitem{\ac{iPD}-stages of \RNum{5} according to the Hoehn \& Yahr scale
  (without medication, i.e. in the OFF stage) \citep{hoehn1967parkinsonism}}\\
& \tabitem{The use of magnetic fields in the MRI examination excludes
  the participation of persons who have electrical devices
  (e.g. cardiac pacemakers, medication pumps, etc.) or metal parts
  (e.g. screws after bone fracture) in or on their bodies.} \\
& \tabitem{Women who are pregnant will not receive MR imaging.} \\
& \tabitem{Subjects who do not want to be informed about possible
  incidental findings are also not allowed to participate in the
  imaging part of the study.}
\\ \midrule

\textbf{Inclusion criteria \ac{PD}-patients' relatives} &
\tabitem{Relatives of patients included in the study according to the abovementioned criteria} \\
& \tabitem{Subjects with the ability to give informed consent} \\
& \tabitem{Subjects with a good knowledge of German}
\\ \midrule

\textbf{Exclusion criteria \ac{iPD}-patients' relatives} &
Relatives who are unable to give informed consent cannot be considered for study participation
\\ \midrule

\textbf{Statistical methods} &
Due to a large number of possible primary endpoints for consideration, a multitude of analyses will be conducted to examine changes and variability over the 20 years. Analyses will include logistic, linear, and longitudinal models for assessing these data, among others. Primary interest will focus on established and long measures of quality-of-life (such as the \ac{PDQ39}) but also of clinical severity such as the \ac{UPDRS}. For the secondary endpoints, clinical, imaging, and biologic verification studies on promising biological markers in study subsets using stored collected samples sould be performed. These studies may vary substantially depending on the type of marker and the data available.
\\ \midrule

\textbf{Statistical test method} & 
According to the primary endpoint and the envisaged number of participants, \ac{ANOVA} will be conducted in order to determine the predictors of decreases in quality of life during the \ac{iPD}-patients' course. Furthermore, studies correlating quality of life to the identified markers and results in imaging should be performed. 
\\ \midrule

\textbf{Sample Size Parameters} & 
This is a longitudinal cohort study where the number of participants is determined by the number of resources available. The number of participants is therefore composed of the expected number of patients in the district and the resources available to the centre (\UKGM).
\\ \bottomrule
\end{tabularx}
\newpage

%%%%%%%%%%%%%%%%%%%%%%%%%%%%%%%%%%%%%%%%%%%%%%%%%%%%%%%%%%%%%%%%%%%%%%%
%%%% 										END OF SYNOPSIS													%%%%
%%%%%%%%%%%%%%%%%%%%%%%%%%%%%%%%%%%%%%%%%%%%%%%%%%%%%%%%%%%%%%%%%%%%%%%

\setstretch{1.5}
\section{Study objectives and endpoints}
\subsection{Study objectives}
The primary aim of the \textsc{HessenKohorte} is to deepen the understanding of the development of quality of life in patients with \ac{iPD} and to identify factors that contribute to a good or poor perceived quality of life in an average German cohort. Furthermore, the study aims to improve the understanding of the impact of the disease on the caregivers and to find out which factors or forms of structural support make these family caregivers more resilient to the burdens associated with caring for a person with \ac{iPD}.

\subsection{Primary study endpoint}
The primary endpoint of the \textsc{HessenKohorte} is the index patients' quality of life. For that a model was developed which measures not only the disease realted aspects of life quality but also additional factors which foster quality of life and thus go beyond healthcare-related issues. This will be ascertained along with established questionnaires such as the \ac{PDQ39}, or the \ac{WHOQoL} and may be related to other characteristics of included patients.

\subsection{Secondary study endpoint}
According to the large number of possible analyses and secondary endpoints for consideration, the authors foresee a substantial
number of analyses to be conducted in order to contemplate changes over time. Yet, a list of a few possible secondary endpoints should be named:
\begin{itemize}
  \item{Changes in motor symptoms of the index patient after up to 20 years}
  \item{Development of non-motor symptoms over up to 20 years}
  \item{Changes of the functional imaging over the observational period} 
\end{itemize}
Irrespective of the unique characteristics of a study on quality of life of people with \ac{iPD}, standard examinations of \ac{iPD}-patients should also be carried out as far as the secondary endpoints are concerned. At this point motor along with non-motor symptoms should be mentioned as the most common symptoms encountered in \ac{iPD} studies. The former constitute the hallmark of the disease, whereas the latter are incresingly recognized as responsible for huge losses in quality of life (Quelle). The extent of motor symptoms are operationalised via part III of the \ac{MDS-UPDRS}. Non-motor symptoms are measured using the \ac{NMSQ} a measure able to capture a wide variety of different aspects with regard to this symptom domain.

% We should mention all questionnaires at this point and in some way

\section{Study design}
The \textsc{HessenKohorte} Study is a prospective 20-year cohort study. The \textsc{HessenKohorte} study aims to recruit patients of all genders suffering from clinically probable \ac{iPD} (n = \SI{1000}{}) along with their relatives in the German region of Hessen. All patients will be recruited from the treated patients in an in- and outpatient settings at the \UKGM between 2022 and 2042. 

\subsection{Scale and duration}
The study will accompany up to \SI{1000}{} patients over at most 20 years to enable a profound insight into the course of the individual patients and their relatives.

\subsection{Justification for study design}
This study is a single-center prospective and longitudinal cohort study to monitor \ac{iPD}-patients development over the course of the disease and particularly to assess their quality of life in a holistic approach. The comparatively large number of subjects will enable a better insight into \ac{PD} with its multifacetted phenotypes

\subsection{Hypotheses}
\label{sec:hypotheses}
The study addresses the folloing main scientific hypothesis:
\begin{enumerate}
  \item Quality of life of the index patient is related to quality of life of the caregivers
  \item Quality of life of the caregivers is related to quality of life of the index patient
  \item Disease progress can be predicted by motor and non-motor symptoms in earlier years
\end{enumerate}
Regarding the first two hypotheses we believe that quality of life of the index \ac{PD} patient and quality of life of her relatives and caregivers are mutually dependent on each other. While there may exist several mediating or moderating variables we think that this basic relationship holds true and will be responsible for a substantial amount of explained variance. With regard to the third hypothesis we think that it is possible to predict the timecourse of symptom development by using information from earlier years. In particular we aim at making individual prognosis more accurate by revealing the nature of the underlying function and the influencing factors.

\subsection{Planned analyses}
Our main method for assessing our hypotheses are linear regression and correlation analyses. In particular we will analyse the correlation between the quality of life of the index patient and of her relatives and caregivers. In order to extend this analysis we will assess different influencing factors as e.g. age, sex, symptom burden (motor and non-motor). Furthermore in an attempt to elucidate the causal form of the interdependence we will analyse how quality of life of patients/caregivers at earlier timepoints relates to quality of life of caregivers/patients at a later point in time.

With regard to our hypothesis concerning the prediction of disease progress we will also primarily rely on linear regression methods. We will try to predict disease symptoms (motor and non-motor) from symptoms at earlier timepoints. While we first will look at a linear relationship we will also consider more complicated models when they are necessary to model disease progress convincingly. Also in this analysis mediating or moderating variables especially from sociodemographic data may play a role and will accordingly be considered.


\section{Subject selection}
\label{sec:study_selection}
\subsection{Study population and Eligibility}
\label{sec:study_population}
Study candidates will be drawn from the patients treated in the Neurology Department of the \UKGM (Marburg site) as either in- or outpatients. Moreover, patients suffering from \ac{PD} may submit a request for participation in the study. The inclusion and exclusion criteria (cf. Section \ref{sec:study_population}) are checked by one of the study physicians, who are responsible for the final decision. Advertising for the study can be found in the form of a flyer, which is available in the Department of Neurology, but also in the form of an Internet, where the project is presented.

\subsection{Inclusion and exclusion criteria \ac{PD}-patients}
\label{sec:inclusion_criteriaIPS}
Subjects who meet all the following inclusion criteria may be given consideration for inclusion in this cohort study, provided no exclusion criteria are met (for both, cf. Table \ref{inclusion_exclusionCriteriaPatients}).

% Table need a caption
\begin{tabularx}{\textwidth}{X | X}
\toprule
\tabitem{Patients suffering from a clinical diagnosis of idiopathic Parkinson's syndrome according to the recent clinical diagnostic criteria \citep{postuma2015mds}} & \tabitem{Patients suffering from a clinical diagnosis of atypic Parkinson's syndrome in a first instance. Patient's enrolled who were later characterized as atypical Parkinson syndroms will not be excluded.} \\
\tabitem{\ac{iPD}-stages of I - IV according to the Hoehn \& Yahr scale (without medication, i.e. in the OFF stage) \citep{hoehn1967parkinsonism}} & \tabitem{\ac{iPD}-stages of V according to the Hoehn \& Yahr scale (without medication, i.e. in the OFF stage) \citep{hoehn1967parkinsonism}} \\
\tabitem{Patients with the ability to provide informed consent. In cases where participants lose their capacity to consent at follow-up visits (e.g., due to dementia, etc.), this participant will only be allowed to continue if a legal representative (proxy, guardian) provides informed consent to further participation on behalf of the participant. In this case, the legal representatives will be provided with a separate consent form.} & \tabitem{The use of magnetic fields in the MRI examination excludes the participation of persons who have electrical devices (e.g. cardiac pacemakers, medication pumps, etc.) or metal parts (e.g. screws after bone fracture) in or on their bodies.}\\
\tabitem{Patients with a good knowledge of German} & \tabitem{Women who are pregnant will not receive \ac{MRI} scans.} \\
& \tabitem{Subjects who do not want to be informed about possible incidental findings are also not allowed to participate in the imaging part of the study.} \\
\bottomrule
\label{tab:inclusion_exclusionCriteriaPatients}
\end{tabularx}

\subsection{Inclusion criteria \ac{PD}-patients' relatives}
\label{sec:inclusion_criteriaREL}
Only if a patient is included, the relatives may be asked for participation in the study. Subjects who agree to take part in the \textsc{HessenKohorte} must meet all the following inclusion and exclusion criteria (cf. Table \ref{tab:inclusion_exclusionCriteriaRelatives}).

% Table need a caption
\begin{tabularx}{\textwidth}{X | X}
\toprule
\tabitem{Relatives of \ac{iPD}-patients included in the study according to the abovementioned criteria (cf. Table \ref{tab:inclusion_exclusionCriteriaPatients})} & \tabitem{Relatives who are unable to give informed consent} \\
\tabitem{Relatives with the ability to give informed consent} &  \\
\tabitem{Relatives with a good knowledge of German} & \\ 
\bottomrule
\label{tab:inclusion_exclusionCriteriaRelatives}
\end{tabularx}

\section{Subject accountability}

\subsection{Point of enrollment}
A subject will be considered enrolled at the time of the study-specific informed consent form (ICF) execution. No study-related procedures or assessments can take place until the ICF is signed.

\subsection{Withdrawal}
All subjects enrolled in the \textsc{HessenKohorte} (including those withdrawn from the clinical study) shall be accounted for and documented. If a subject withdraws from the clinical investigation, the reasons shall be reported.

Reasons for withdrawal include but are not limited to:
\begin{itemize}
  \item subject or relative choice to withdraw consent
  \item lost to follow-up
  \item pregnancy*
  \item implantation of electrical devices or metal parts in or on the body *
\end{itemize}

* Only the MR-imaging will be discontinued during pregnancy or from the moment of an implantation onwards.

Subjects may of course withdraw at any time, with or without reason, and without prejudice to further treatment. All applicable case report forms \ac{CRF} up to the point of subject withdrawal and an ``End of Study'' form must be completed. Any subject deemed ``lost to follow-up'' should have a minimum of three documented attempts to contact him/her prior to completion of the ``End of Study'' form. Additional study data may no longer be collected after the point at which a subject has been withdrawn from the study or withdraws consent, for whatever reason. Data collected up to the point of subject withdrawal may be used. Subjects withdrawn after completing the implant procedure will not be replaced 
%% Zu Klären End of Study Form? Sollen ausgeschiedene ersetzt werden?.

\subsection{Lost to follow-up}
% Was passiert im Falle einer Scheidung/des Tods eines Angehörigen bzw. wenn ein anderer Angehöriger teilnehmen möchte?
\subsection{Subject status and classification}
A subject will be considered enrolled in this study at the time of the study-specific \ac{ICF} execution.

\subsection{Enrolment control}
The overall enrollment in the study will be capped at 1000 participants.

\subsection{End-of-study definition}
The study is considered complete when 20 years from the first enrolment are over.

\section{Study methods}
\subsection{Data collection}
The data collection schedule is shown in Table \ref{}
\newpage
%\begin{landscape}
\begin{table}[H]
\caption{Data Collection Schedule for \ac{PD}-patients enrolled in the \textsc{HessenKohorte}}
\begin{tabularx}{1\textwidth}{@{}X *{6}{C}@{}}
\toprule
\textbf{Visit} 				& \textbf{Screening} 	& \textbf{Baseline visit} 	& \textbf{Half-year visit} 	& \textbf{Year 1,2,3,4, ..., 20 Visit} 	& \textbf{Unscheduled Visit} 	\\
\cmidrule{2-6}
Informed Consent Process 	& X 					&  						& 						& 								& 							\\
Eligibility Criteria			& \multicolumn{2}{c}{X}							& 						& 								& 							\\
Subject Demographics 		& \multicolumn{2}{c}{X}							& X 						& X 								& 							\\
\ac{MDS-UPDRS} 			& \multicolumn{2}{c}{X}							&  						& X 								& X*							\\
\ac{NMSQ}				& \multicolumn{2}{c}{X}							&						& X								&							\\
\ac{CHAPO-PD}			& \multicolumn{2}{c}{X}							&						& X								&							\\
Hair sample				& \multicolumn{2}{c}{X}							&						& X								&							\\
Saliva sample				& \multicolumn{2}{c}{X}							&						& X								&							\\
Blood  sample			& \multicolumn{2}{c}{X}							&						& X								&							\\
Stool sample				& \multicolumn{2}{c}{X}							&						& X								&							\\
\bottomrule
\multicolumn{6}{l}{\footnotesize{*may be ascertained and entered into database}}
\end{tabularx}
\end{table}
\newpage

\begin{table}[H]
\caption{Data Collection Schedule for patients' relatives enrolled in the \textsc{HessenKohorte}}
\begin{tabularx}{1\textwidth}{@{}X *{6}{C}@{}}
\toprule
\textbf{Visit} 				& \textbf{Screening} 	& \textbf{Baseline visit} 	& \textbf{Half-year visit} 	& \textbf{Year 1,2,3,4,5, ..., 20 Visit} 	& \textbf{Unscheduled Visit} 	\\
\cmidrule{2-6}
Informed Consent Process 	& X 					&  						& 						& 								& 							\\
Eligibility Criteria			& \multicolumn{2}{c}{X}							& 						& 								& 							\\
Subject Demographics 		& \multicolumn{2}{c}{X}							& X 						& X 								& 							\\

Hair sample				& \multicolumn{2}{c}{X}							&						& X								&							\\
Saliva sample				& \multicolumn{2}{c}{X}							&						& X								&							\\
Blood  sample			& \multicolumn{2}{c}{X}							&						& X								&							\\
Stool sample				& \multicolumn{2}{c}{X}							&						& X								&							\\
\bottomrule
\multicolumn{6}{l}{\footnotesize{*may be ascertained and entered into database}}
\end{tabularx}
\end{table}
%\end{landscape}

\subsection{Candidate Screening}
\label{subsec:screening}
Subjects will be screened for participation in the study based on study Inclusion and exclusion criteria as listed in Section \ref{sec:study_selection}. Subjects who have provided informed consent and who have been determined to not meet all eligibility requirements will be withdrawn.

\subsection{Informed consent}
Written informed consent must be obtained from potential study candidates and enrollment is only valid, after subjects sign and date the \ac{ICF}.
\begin{itemize}
\item Subjects will be asked to sign the \ac{ICF} before study-specific tests or procedures are performed;
\item The idea of the study must be explained, and subjects must be given the time and opportunity to ask questions and have those questions answered to their satisfaction.
\item The \ac{ICF} is study specific and has been approved by the \ac{EC}.
\item Written informed consent must be recorded appropriately by means of the subject’s dated signature.
\end{itemize}

\subsection{Questionnaires}
\label{subsec:questionnaires}
\subsubsection{\acl{CHAPO-PD}}

\subsubsection{\acl{MDS-UPDRS}}
The \ac{MDS-UPDRS} evaluates various aspects of \ac{PD}-patients, including non-motor and motor symptoms. It consists of four parts:
\begin{itemize}
\item Part I: Experiences of daily living (non-motor symptoms), including 13 items.
\begin{itemize}
\item A: Behavioral problems of the patient, as evaluated by the examiner.
\item B: Part on non-motor symptoms completed by the patient, with the assistance of a caregiver if necessary, but independent of the investigator.
\end{itemize}
\item Part II: Experiences of daily living (motor aspects) with 13 items. This part is also a self-report questionnaire to be completed by the patient, with the assistance of a caregiver if necessary, but independent of the investigator.
\item Part III: Motor examination with 18 items. All instructions are read to the patient by the examiner or demonstrated directly, so that this part is completed by the examiner.
\item Part IV: Motor Complications with 6 items. This part contains instructions for the examiner and also instructions to be read to the patient. It combines patient-related information with clinical observations and assessments by the examiner.
\end{itemize}

\subsubsection{\acl{MoCa} (\acs{MoCa})}
\acl{MoCa} is a screening tool that can quickly identify hints on cognitive decline according to mild cognitive impairment or dementia. The entire test consists of approximately 10-minutes of questions around different domains of cognitiion. The 30 questions test cognitive abilities such as memory, language production , contextual thinking, attention and concentration, behavior, arithmetic, temporal and spatial orientation, and the ability to recognize complex shapes and patterns. The test is validated is extensively applied in clinical routine. 
% TODO add details from https://www.sralab.org/rehabilitation-measures/montreal-cognitive-assessment

\subsubsection{\acl{NMSQ}, (\acs{NMSQ})}
The \ac{NMSQ} is a 30-item rater-based scale designed to assess a broad
spectrum of non-motor symptoms in patients with Parkinson's disease
(PD).  (PD). The \ac{NMSQ} measures the severity and frequency of non-motor
symptoms across nine dimensions.

\subsubsection{\acl{BDI} (\acs{BDI})}
The \ac{BDI} is a questionnaire which aims at assessing the severity of depressive symptoms in case depression exists. It is not intended to assess depression per se, but only its severity. Hence, it cannot be used as a screening method in the normal population so taht other alternatives should be contemplated. The applied second version of the \ac{BDI} consists of  21 questions which are supposed to be evaluated for the previous two weeks. 

Scores:
\begin{itemize}\itemsep2pt
\item 0–12: no depressive symptoms or clinically inapparent
\item 13–19: mild depressive syndrome
\item 20–28: moderate depressive syndrome
\item $>$ 29 severe depressive syndrome
\end{itemize}

\subsubsection{\acl{CBI} (\acs{CBI})}
The \ac{CBI} is a questionnaire for measring personal burnout, work-related burnout, and client-related burnout with a very high internal reliability. It has shown to correlate with future sickness absence, sleep problems, the use of pain-killers, and intention to quit the job. 

\subsubsection{\acl{CISS} (\acs{CISS})}
The \ac{CISS} is an instrument for the assessment of coping style. It assesses three different styles: task-oriented coping, emotion-riented coping and avoidance-oriented coping which in turn can be subdivided into social distraction-oriented coping and more general distraction-oriented coping.

\subsubsection{\acl{MFI-20}}
Fatigue is a state where a person experiences a reduced level of energy in daily life activities. The \ac{MFI-20} is an instrument which measures fatigue via 20 different items on five subscales: general fatigue, physical fatigue, mental fatigue, reduced motivation and reduced activity.

\subsubsection{\acl{PDCB} (\acs{PDCB})}
The \ac{PDCB} is a questionnaire specifically aiming at caregiver burden for caregivers of patients with Parkinson's disease (PD). It assesses the caregivers burden on the seven subscales of physical burden, sleep disruption, patient symptoms, responsibilities, patient medications, social burden as well as patient and self-relationship. Furthermore a global measure of caregiver burden is given. 

\subsubsection{\acl{PHQ}}
% TODO

\subsubsection{\acl{PSS} (\acs{PSS})}
The perceived stress scale is an instrument for the assessment of the subjective stressfulness of different life events across the last month. A global score is obtained which can be used to subdivide subjects into groups of low stress (0-13 points), moderate stress (14-26points) and high stress (27-40 points).

\subsubsection{\acl{WHOQoL} (\acs{WHOQoL})}
The \ac{WHOQoL} is a detailed, standardized measure of ones Quality of Life (QoL) which is available in many languages and puts an emphasis on cross-cultural comparability. It features 100 questions which can be assessed with six domain scores, 24 \emph{facet} scores as well as a global QoL \emph{facet} score. The domain scores describe QoL on the following dimension: physical, psychological, level of independence, social relationships, environment, and spirituality.

\subsubsection{\acl{ZBI-22} (\acs{ZBI-22})}
The \ac{ZBI-22} is a questionnaire assessing the burden of a caregiver caring for another person on a 22 item scale. It is one of the moust commonly used tools to assess this type of burden. % TODO more ....

\subsection{Baseline visit \ac{PD}-patients}
All potential candidates will undergo screening procedures as listed in Section \ref{subsec:screening} to determine their eligibility in the study. Subjects may neither have to be on stable anti-parkinsonian medications prior to informed consent nor have to be regularly treated at the \UKGM. Those subjects who meet all inclusion criteria and none of the exclusion criteria (cf. \ref{sec:study_selection}) may be enrolled. The baseline visit may occur anytime within the screening period and will serve as the final  determination of eligibility in the study. 

The following data from questionnaires should be collected from patients:
\begin{itemize}
\item General Assessments
\begin{itemize}
\item Demographic data and personal information
\item Medication schedule
\end{itemize}
\item \ac{CHAPO-PD}
\item \ac{NMSQ}
\end{itemize}


\subsection{Half year visit \ac{PD}-patients ($\pm$ 100 days)}
% Wirklich 100 Tage oder wie sind die Daten definiert?

\subsection{Annual visit \ac{PD}-patients ($\pm$ 100 days)}
% Wirklich 100 Tage oder wie sind die Daten definiert?

\subsection{Baseline visit relatives}
This study is intended as inclusion of diades of patients and relativesAll potential candidates will undergo screening procedures as listed in Section \ref{subsec:screening} to determine their eligibility in the study. Subjects may neither have to be on stable anti-parkinsonian medications prior to informed consent nor have to be regularly treated at the \UKGM. Those subjects who meet all inclusion criteria and none of the exclusion criteria (cf. \ref{sec:study_selection}) may be enrolled. The baseline visit may occur anytime within the screening period and will serve as the final  determination of eligibility in the study. 

For the relatives, the following data from questionnaires should be collected:
\begin{itemize}
\item General Assessments
\begin{itemize}
\item Demographic data and personal information
\item Relationship to patients
\item Experiencing respect in the patient-family relationship
\end{itemize}
\item \acl{BDI}, (part II)
\item \ac{CBI}
\item \ac{CISS}
\item \ac{MFI-20}
\item \ac{MoCa}
\item \ac{PDCB}
\item \ac{PHQ}
\item \ac{PSS}
\item \ac{WHOQoL}
\item \ac{ZBI-22}
\end{itemize}

\subsection{Half year visit relatives ($\pm$ 100 days)}

\subsection{Annual visit relatives ($\pm$ 100 days)}
% Welchen ``Spielraum'' ermöglichen wir den Patient:innen für die nächste Messung?

\subsection{\ac{MRI}}
Every \ac{PD}-patient will receive MR-imaging if no contraindication exists and at the request of the respective patient. With the aim of producing the greatest possible synergistic effects with other large studies at the centre and to ensure a high quality of the sequences, the programme to be run was based on the PPMI study (\url{https://www.ppmi-info.org/}). Further details are disclosed below.

\subsubsection{Overview of MR-imaging}
\begin{table}[h]
\caption{Overview on the \ac{MRI}-sequences in use during the \textsc{HessenKohorte}*}
\begin{tabularx}{1\textwidth}{@{}X *{1}{C}@{}}
\toprule
\textbf{Sequence Name} 						& \textbf{Series Description }	\\
\midrule
T1-weighted, 3D volumetric sequence 			& 3D T1-weighted 		\\
2D Gradient-echo T2*-weighted EPI (BOLD) 		& rsfMRI\_RL 			\\
Repeat 2D Gradient-echo T2*-weighted EPI (BOLD) 	& rsfMRI\_LR 			\\
NM-MT 										& 2D GRE-MT 			\\
DTI 											& DTI\_RL 			\\
Repeat DTI 									& DTI\_LR 			\\
3D T2 FLAIR 									& 3D T2 FLAIR 		\\
\bottomrule
\multicolumn{2}{l}{\footnotesize{*protocol is identical to the one used by the \ac{PPMI}-study}}
\end{tabularx}
\end{table}

\subsubsection{Procedure of the imaging}
% Werden NUR die Patienten gemessen oder auch die Angehörigen?
Participants should be positioned comfortably and correctly to minimize motion during the scan. Furthermore, technicians will be instructed to comply with the following:
\begin{itemize}
\item Participant should be informed about the total acquisition time and positioned for maximum comfort.
\item Subjects must be positioned comfortably and supine in the head coil to minimize any motion during the scan.
\item Proper back support, and support under the knees will ensure greater comfort, and lead to less motion in the scan.
\item There should be no left-right or ear-to-shoulder head tilt, and the participant’s neck should not be hyper- extended or retracted.
\item Subject's head should be centered in the head coil using the nasion (see example to the right) as an anatomical landmark. Ensure the participant is high enough in the coil to avoid loss of signal at the inferior aspects of the brain.
\item Immobilization devices, such as velcro straps, or foam padding should be used to reduce motion.
\item The positioning lasers should be used to send the nasion to the magnets isocenter.
\end{itemize}
If a participant’s neck length is such that it does not permit proper positioning in the head coil, please document this on the \ac{MRI} Acquisition Document along with any other pertinent information regarding the participants scanning session.
% Brauchen wir ein Acquisition Document?

\newcolumntype{s}{>{\hsize=.45\hsize}X}
\subsubsection{T1-weighted, 3D volumetric sequence}
\begin{table}[H]
\caption{Details on T1-weighted \ac{MRI}-sequence}
\begin{tabularx}{\linewidth}{@{} s | X @{}}
\toprule
\multicolumn{2}{p{\dimexpr\linewidth-2\tabcolsep-2\arrayrulewidth}}{\textbf{T1-weighted, 3D volumetric \ac{MRI}-sequence during the \textsc{HessenKohorte}, e.g. \ac{MP-RAGE}, \ac{IR-FSPGR}}} \\
%\multicolumn{2}{l}{} \\
\midrule
Series description 								& 3D T1-weighted 											\\
Plane	 									& Sagittal 												\\
Slice thickness (mm) 							& 1.0 (slice thickness must remain consistent across timepoints) 	\\
Number of slices 								& 192 (slice thicksness may be adjusted to 1.2 mm to cover brain iff absolutely necessary. No adjustments of number of slices) 			\\
Voxel size (mm) 								& 1.0*1.0 mm in plane resolution \\
Phase encode direction 						& Anterior Posterior (AP) 			\\
Matrix										& 256 $\times$ 256 (the use of interpolation, zero-filling or a ZIP factor is not permitted)\\
TR/TE/FA/ other parameters 					& Will be defined by Invicro according to the scanner\\
FoV		 									& 256 mm (full FoV required, no rectangular FoV)\\
Scan time 									& $\sim$ 7 min\\
Further explanations 							& The FOV must include the entire brain anatomy including the vertex, cerebellum and pons. Slices should be oblique sagittal, angled along the longitudinal fissure on both the axial and coronal localizers. To avoid artifacts, position the participant such that there is sufficient empty space around the head: approximately 1.5 cm of air or more above the top of the head, and leave 3 - 4 blank slices on either side of the head. Avoid nose ghosting.\\
\bottomrule
\multicolumn{2}{l}{\footnotesize{*protocol is identical to the one used by the \ac{PPMI}-study}}
\end{tabularx}
\end{table}

\subsubsection{2D Gradient-echo T2*-weighted EPI}
\begin{table}[H]
\caption{Details on T2-weighted \ac{MRI}-sequence}
\begin{tabularx}{\linewidth}{@{} s | X @{}}
\toprule
\multicolumn{2}{p{\dimexpr\linewidth-2\tabcolsep-2\arrayrulewidth}}{\textbf{2D-Gradient-echo T2*-weighted \ac{EPI} (e.g., ep2d\_BOLD)}} \\
\midrule                                                                                                                                                                                                                                                                                                                                                                                                                                                                                                                                                                                                                                                                                                                          
Series Description                                				& rsfMRI\_RL \\
Plane                                             					& Axial Oblique, plane parallel to AC-PC line \\
Slice thickness (mm)                              				& 3.5 with no gap \\
Number of Slices                                  				& $\sim$40 \\
Phase encode dir.                                 				& R \textgreater{}\textgreater L\\
Matrix                                            					& 64 $\times$ 64 \\
FOV                                               						& 224 $\times$ 224 mm \\
Repetition Time (ms)                              				& 2500 \\
Echo Time (ms)                                    				& 30 \\
Flip angle                                        					& 80 \\
Slice order                                       					& Interleaved \\
Number of measurements                            			& 240 (10 min total scan time) \\
In-plane acceleration                             				& GRAPPA or SENSE (factor of 2) \\
Instructions                                      					& Keep the eyes open and remain still \\
Scan Time                                         					& $\sim$10 min \\
Further explanations                              				& Please instruct the participant to keep their eyes open during the entire scan. You can instruct them to focus on a point on the mirror or scanner. Check with the participant immediately after the scan to verify they kept their eyes open and did not fall asleep. No audio or video presentation should be made during the scan.Position the axial resting state fMRI slices along the AC-PC plane with care that there is one slice above the vertex, and then cover the rest of the brain and as much of the cerebellum as possible with the remaining slices. The slices should be centered in the axial plane to prevent aliasing in the Anterior/Posterior direction (see Figure 4 ??). \ac{TR}/\ac{TE} should not be changed. \\
\bottomrule
\end{tabularx}
\end{table}

\subsubsection{REPEAT 2D Gradient-echo T2*-weighted EPI}
\begin{table}[H]
\caption{Details on REPEAT T2-weighted \ac{MRI}-sequence}
\begin{tabularx}{\linewidth}{@{} s | X @{}}
\toprule
\multicolumn{2}{p{\dimexpr\linewidth-2\tabcolsep-2\arrayrulewidth}}{\textbf{REPEAT 2D Gradient-echo T2*-weighted \ac{EPI}}} \\
\midrule                                                                                                                                                                                                                                                                                                                                                                                                                                                                                                                                                                                                                                                                                                                          
Series Description                                                                	& rsfMRI\_LR                                  \\
Plane                                                                                      	& Axial Oblique, plane parallel to AC-PC line \\
Slice thickness (mm)                                                          	& 3.5 with no gap                             \\
Number of Slices                                                      		& $\sim$40                                    \\
Phase encode dir.                                                                 	& L \textgreater{}\textgreater R              \\
Matrix                                                                                     	& 64x64                                       \\
FOV                                                                                        	& 224 x 224 mm                                \\
Repetition Time (ms)                                                             & 2500                                        \\
Echo Time (ms)                                                                        & 30                                          \\
Flip angle                                                                                 	& 80                                          \\
Slice order 									& Interleaved                                 \\
Number of measurements                                                  & 10 (25 sec total scan time)                 \\
In-plane acceleration                                                             & GRAPPA or SENSE (factor of 2)               \\
Instructions									& Keep the eyes open and remain still         \\
Further explanations                                                             & Repeat the above scan with the phase encoding direction updated to L >> R, and the number of measurements updated to “10”. All other parameters should be held constant. Recommended imaging parameters for the repeat resting state fMRI sequence can be referenced in Table 6.                                            \\
\bottomrule
\end{tabularx}
\end{table}

\subsubsection{2D Gradient recalled echo with MT preparation}
\begin{table}[H]
\caption{Details on REPEAT T2-weighted \ac{MRI}-sequence}
\begin{tabularx}{\linewidth}{@{} s | X @{}}
\toprule
\multicolumn{2}{p{\dimexpr\linewidth-2\tabcolsep-2\arrayrulewidth}}{\textbf{2D-Gradient-echo T2*-weighted \ac{EPI} (e.g., ep2d\_BOLD)}} \\
\midrule                                                                                                                                                                                                                                                                                                                                                                                                                                                                                                                                                                                                                                                                                                                          
Series Description                                				& rsfMRI\_RL                                  \\
Plane                                             					& Axial Oblique, plane parallel to AC-PC line \\
Slice thickness (mm)                              				& 3.5 with no gap                             \\
Number of Slices                                  				& $\sim$40                                    \\
Phase encode dir.                                 				& R \textgreater{}\textgreater L              \\
Matrix                                            					& 64x64                                       \\
FOV                                               						& 224 x 224 mm                                \\
Repetition Time (ms)                              				& 2500                                        \\
Echo Time (ms)                                    				& 30                                          \\
Flip angle                                        					& 80                                          \\
Slice order                                       					& Interleaved                                 \\
Number of measurements                            			& 240 (10 min total scan time)                \\
In-plane acceleration                             				& GRAPPA or SENSE (factor of 2)               \\
Instructions                                      					& Keep the eyes open and remain still         \\
Scan Time                                         					& $\sim$10 minutes                            \\
Further explanations                              				& Please instruct the participant to keep their eyes open during the entire scan. You can instruct them to focus on a point on the mirror or scanner. Check with the participant immediately after the scan to verify they kept their eyes open and did not fall asleep. No audio or video presentation should be made during the scan.                                           
\end{tabularx}
\end{table}


\subsubsection{2D Diffusion-weighted EPI}
\begin{table}[H]
\caption{Details on 2D Diffusion-weighted EPI}
\begin{tabularx}{\linewidth}{@{} s | X @{}}
\toprule
\multicolumn{2}{p{\dimexpr\linewidth-2\tabcolsep-2\arrayrulewidth}}{\textbf{2D Gradient-echo T2*-weighted \ac{EPI} (eg ep2d\_BOLD)}} \\
\midrule                                                                                                                                                                                                                                                                                                                                                                                                                                                                                                                                                                                                                                                                                                                          
Series Description        							& \ac{DTI}\_RL (and \ac{DTI}\_LR for the repeated scan with reverse PE)                          \\
Plane                    						 		& Straight Axial                                                                       \\
Slice thickness (mm)      							& 2.0 with no gap                                                                      \\
Number of Slices          							& $\sim$80                                                                             \\
Phase encode dir.         							& R \textgreater{}\textgreater L                                                       \\
Matrix                    								& 128x128*                                                                             \\
FOV                       								& 256x256 mm                                                                           \\
Repetition Time (ms)      						& $\sim$10000                                                                          \\
Echo Time (ms)            							& $\sim$80                                                                             \\
Flip angle                								& 90                                                                                   \\
Slice order               								& Interleaved                                                                          \\
Number of directions      						& 32                                                                                   \\
b-VALUE                   								& 0 and 1000 s/mm2 (B=0 images interleaved throughout if possible in product sequence) \\
Instructions              							& Keep still                                                                           \\
Scan Time                 								& $\sim$8 minutes                                                                      \\
Further explanations      						& Please instruct the participant to keep still during the entire scan. \ac{DTI} should be acquired with 32 directions. Slices should cover top of the brain down to base of cerebellum. Two sequences with reversed phase encoding direction should be acquired in full to correct for susceptibility induced distortions. If acquiring a phantom scan, only one sequence with reverse phase encoding direction should be acquired                                                                                     
\end{tabularx}
\end{table}

\subsubsection{3D T2 \ac{FLAIR} Sequence}
\begin{table}[H]
\caption{Details on T2-weighted \ac{FLAIR} Sequence}
\begin{tabularx}{\linewidth}{@{} s | X @{}}
\toprule
\multicolumn{2}{p{\dimexpr\linewidth-2\tabcolsep-2\arrayrulewidth}}{\textbf{3D T2 \ac{FLAIR} Sequence}} \\
\midrule                                                                                                                                                                                                                                                                                                                                                                                                                                                                                                                                                                                                                                                                                                                          
Series Description        							& 3D T2 FLAIR                                                                               \\
Plane                   	 		 					& Sagittal                                                                                  \\
Slice thickness (mm)      							& 1.0 – 1.2 (slice thickness must remain consistent)                                        \\
Number of slices          							& 192 (please adjust slice thickness up to 1.2 mm to cover brain, not the number of slices) \\
Voxel size (mm)           							& 1.0*1.0 mm in plane resolution                                                            \\
Phase encode dir.         							& Anterior-Posterior (AP)                                                                   \\
Matrix                    								& 256 x 256 (the use of interpolation, zero-filling or a ZIP factor is not permitted)       \\
TR/TE/FA/other parameters 					& Will be defined by Invicro according to the scanner                                       \\
FOV                       								& 256 mm (full FOV required, no rectangular FOV)                                            \\
Scan Time                 								& $\sim$7 minutes                                                                           \\
Further explanations      						&  The FOV must include the entire brain anatomy including the vertex, cerebellum and pons. To avoid artifacts, position the participant such that there is sufficient empty space around the head: approximately 1.5 cm of air or more above the top of the head, and leave 1 - 2 blank slices on top of the head. Avoid nose ghosting.                                                                                        
\end{tabularx}
\end{table}

\subsection{Biosamples}
\subsubsection{Hair}
\subsubsection{Saliva}
\subsubsection{Urine}
\subsubsection{Blood}
\subsubsection{Stool}
% SOP für die Untersuchungen durch SJ/EM

\section{Statistical considerations}


\section{Data management}
% wer soll das schreiben? Tim, David, Urs?

\section{Amendments}
In case of protocol changes possibly affecting the rights, safety or welfare of any subjects or scientific integrity of the data, a protocol amendment will be completed. Appropriate approvals (especially from the \ac{EC}) of the revised protocol must be obtained prior to its implementation.

\section{Compliance}
\subsection{Statement of Compliance}
This study will be conducted in accordance with ICH-GCP and with the ethical principles originating in the Declaration of Helsinki. 

\subsection{Investigator responsibilities}

\subsubsection{Delegation of responsibilities}
When specific tasks are delegated, the Principal Investigator is responsible for providing appropriate training if necessary and adequate supervision of those to whom tasks are delegated. The investigator is accountable for regulatory violations resulting from failure to adequately supervise the conduct of the clinical study. 

\subsection{Ethics committee}
The investigational site has obtained the approval of the local ethics commitee for the clinical investigation. A copy of the written approval of the protocol can be found in the Appendix (cf. chapter \ref{}). Any amendment to the protocol will require review and approval by the ethics commitee before any changes are implemented to the study. Besides, all changes to the \ac{ICF} will have to be approved, as well. In case of an extension of the study to further centers, an ethics approval must be obtained by the respective ethics commitee. 

\section{Monitoring}
% Wie gehen wir mit AE/SAE um? Braucht man das überhaupt?

\section{Potential Risks and Benefits}

\subsection{Anticipated Adverse Events}
Due to the nature of the study as a longitudinal observational study
we won't consider most medial events and emergencies as adverse
events. Falls, infektions and even death occur naturally in the course
of \ac{PD} and the aim of this study is merely to document these
events in order to learn more about the natural disease
course. Adverse events in the sense of this study are only those
events which wouldn't have happened without study
participations. These are exclusively complications arising from the
sampling of data. The risks associated with these procedures are
outlined below.

\subsection{Risks associated with the study participation}
No particular medical risks are associated with participating in the
study itself since all participants have access to the standard of
care treatment of their condition. Risks associated with the study
participation thus only arise from sampling of the data, which will be
discussed below.

\subsection{Risks associated with the sampling of biodata}
While the sampling of stool, hair, urine and saliva is also not
assiciated with a particular risk the sampling of blood carries the
usual, rather minor risks of numbness due to nerve injury, infection
at the site of venous puncture, hematoma or fainting. Participants
will be educated about these risks and the risk will be minimized by
only employing experienced personnel for the collection of the samples
and prefereably collecting the samples while participants are in a
supine position.

\subsection{Risks associated with the \ac{MRI}}
\ac{MRI} is a radiologic method which does not expose the subject to
dangerous X-ray radiation and thus is in general not associated with
many risks. The main risk arises from participants bringing metal
items into the \ac{MRI} scanner which is dangerous due to the strong
magnetic field inside the device. Those items may either be medical or
aesthetic implants, remnants of former accidents or war experiences or
may accidentially be kept e.g. in the pocket before entering the
\ac{MRI} device. During our study every participant will be thorouhgly
educated with regard to this risk before each \ac{MRI}
measurement. Other risks arise due to the loud noise and the narrow
space in the \ac{MRI} scanner. With regard to the former our
participants will receive medical grade ear protection in order to
avoid any hearing loss. This is a well established procedure and
blocks the surrounding noise effectively down to a harmless
amplitude. With regard to psychological problems due to narrow space
we will screen patients before the measurement for any prior signs of
claustrophobia. During the measurements participants may stop the
measurement at any time via a \emph{panic button} handed over to them
as soon as they enter the device.

\section{Safety Reporting}

\section{Informed consent}
All participants will only be included within the study after reading
and signing the informed consent form. The forms for patient
participants and relatives can be found in the appendix of this
document (\ref{sec:icf_patient} and \ref{sec:icf_relative}).

\section{Suspension or termination of the study}

\section{Study registration and Results}

\section{Bibliography}
