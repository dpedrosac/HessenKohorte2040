\begin{center}
  {\Huge SOP Biobank -- Hessenkohorte 2040}
  
  Version 1.1 vom 17.07.2022 
\end{center}

\vspace*{1cm}

\begin{tabular}{@{}p{0.4\textwidth}l}
  erstellt von: Urs Kleinholdermann & am 17.07.2022 \\
  geprüft von: & am \\
  freigegeben von: & am \\
\end{tabular}

\vspace*{2cm}

\noindent{\Large\textbf{1. Ziel und Zweck}}

Beschreibung der Probensammlung und des down-stream-processings in der Biobank im Rahmen der longitudinalen Hessenkohorte Morbus Parkinson. \\


\noindent{\Large\textbf{2. Verbrauchsmaterial}}
\begin{itemize}
  \item Blutentnahme 	
    \begin{itemize}
      \item 1 X 4,6 ml EDTA (Blutbild)
      \item 2 x 9ml EDTA Sarstedt K2 ref. 02.1333.001 
      \item 1 x 8ml CPT (Sodium Citrate) ref. BD 362782
      \item 1 x PAXgene ref. BD 762165
      \item 1 x 15ml Falcon tube konisch 
      \item10 x 1 ml Fluid X tubes 96 ref. Brooks 68-1001-11 
    \end{itemize}
  \item Mittelstrahlurin
    \begin{itemize}
      \item 1 x 20ml urine sample 
      \item 2 x 10ml conical tube Sarstedt
      \item 20 x 1 ml Fluid X tubes 96 ref. Brooks 68-1001-11
    \end{itemize}
  \item Speichel
    \begin{itemize}
      \item 1 x Invitek 1035212200 SalivaGene Collection Module II
      \item 1 x Salivette Sarstedt Art.-Nr. 51.1534.500
    \end{itemize}
\end{itemize}

\noindent{\Large\textbf{3. Ablauf vor der Visite}}
\begin{itemize}
  \item \textbf{Checkliste:}
    Die Biobank stellt eine Checkliste bereit, die als Laufzettel für
    jede Probenahme diese in der Klinik für die Biobank dokumentiert.
  \item \textbf{Wochenplan:}
    Die Klinik sendet vor Wochenbeginn einen Probeneinsendungsplan per
    E-Mail an die Biobank. Änderungen werden per-Mail oder Telephon
    mitgeteilt.
  \item \textbf{Materialkontrolle:}
    Das Studienteam prüft wöchentlich den Materialbestand fordert bei
    Bedarf rechtzeitig entsprechende Materialien an.
\end{itemize}

\noindent{\Large\textbf{4. Probenentnahme}}
\begin{itemize}
  \item \textbf{Blut}
    \begin{itemize}
      \item Material:
        \begin{itemize}
          \item 1 X 4,6 ml EDTA Blutbild
          \item 2 x 9 ml EDTA DNA-Extraktion
          \item 1 x 8 ml CPT PBMC/Buffy Coat
          \item 1 x PAXgene	Transcriptomics
        \end{itemize}
      \item Die Blutentnahme soll in der oben angegebenen Reihenfolge
        vorgenommen werden.
      \item \textbf{Achtung!} Alle Proben werden umgehend in das
        Biobanklabor Klinikgebäude Ebene -3/ Raum 43290 transportiert
        und dort weiterverarbeitet. Der Eingang der Proben wird auf
        der Checkliste vermerkt.
      \item Die EDTA-Probe zu 4,6 ml wird in das Zentrallabor zur
        Bestimmung des Blutbildes transportiert.
      \item Wichtig ist die Berücksichtigung der Begleitschreiben von
        PAXgene sowie CPT Gefäße.
    \end{itemize}
  \item \textbf{Urin}
    \begin{itemize}
      \item 2 x 10 ml Urinröhrchen
    \end{itemize}
  \item \textbf{Speichel}
    \begin{itemize}
      \item 2 x Salivette
      \item 1 x SalivaGene Collection Module II
    \end{itemize}
\end{itemize}

\noindent{\Large\textbf{5. Prä-analytisches Liquid Handling in der Biobank}}
\begin{itemize}
  \item \textbf{3 x 9 ml EDTA}
    \begin{itemize}
      \item Die beiden Röhrchen werden zur Extraktion von DNA zu einem
        späteren Zeitpunkt eingesetzt. Das Vollblut wird in 5 ml
        Aliquots in die entsprechenden Sekundärröhrchen pipettiert und
        diese bei $-80^{\circ}C$ gelagert. Die DNA-Extraktion erfolgt zu einem
        späteren Zeitpunkt (max. Lagerzeit 12Mon.) im Institut für
        Humangenetik, Marburg.
      \item Das dritte Röhrchen dient der Plasmagewinnung und wird
        entsprechend der SOP Plasma-CBBMR prozessiert.
      \item Abweichungen werden dokumentiert.
    \end{itemize}
  \item \textbf{1 x CPT}
    \begin{itemize}
      \item Das CPT-Röhrchen dient der Gewinnung von PBMC aus Buffy
        Coat und wird nach der SOP CPT-CBBMR prozessiert. Bei Einsatz
        einer anderen Isolationsmethode für PMBC kann statt der
        CPT-Röhrchen auch ein EDTA-K2-Röhrchen zur Blutentnahme
        verwendet werden.
    \end{itemize}
  \item \textbf{PAX-Gene}
    \begin{itemize}
      \item Das PAX-Gene-Röhrchen dient der stabilisierten Gewinnung
        von RNA zur Transkriptomanalyse und wird entsprechend der SOP
        CPT-CBBMR behandelt.
    \end{itemize}
  \item \textbf{Mittelstrahlurin}
    \begin{itemize}
      \item Der Patient wird gebeten, ml frischen Mittelstrahlurin im
        ausgegebenen Behälter bereitzustellen.
      \item Noch in der Klinik wird die Probe auf Eis gelagert und in
        das Biobanklabor transportiert. Dort werden 20 ml des
        gekühlten Urins abgenommen, in 2 x 10 ml Starstedt-Urintubes
        überführt und bei 400g, $+4^{\circ}C$, 5min, ohne Bremse zentrifugiert.
      \item Die Überstände in Aliquots á 0,5 ml in entsprechende
        FluidX-Röhrchen aliquotiert und bei $-80^{\circ}C$ gelagert.
      \item Die Pellets werden in je 1,250 ml RNA-Cell Protect-Medium
        aufgenommen. und bei $-80^{\circ}C$ gelagert.
    \end{itemize}
  \item \textbf{Speichelsammlung für die Metabolomics}
    \begin{itemize}
      \item Die Sammlung des Speichels erfolgt mittels der Salivette
        (Sarstedt) zur Sammlung von unstimulated whole-mouth saliva
        (UWMS). Es werden zwei Röhrchen befüllt. Die Entnahme mittels
        “Kagummi erfolgt nach Beilagenvorschrift. Sobald das Röhrchen
        gefüllt ist, wird es auf Eis zur weiteren Laborbearbeitung
        gelagert.
      \item Das Röhrchen wir nach Vorgabe samt Kaugummi zentrifugiert
        (1200g; $+4^{\circ}C$, 20 min. ohne Bremse).
      \item Nach der Zentrifugation wird das Kaugummi entnommen und
        verworfen, die verbliebene Flüssigkeit mit der Pipette
        homogenisiert
      \item Aus der homogenisierten Flüssigkeit werden Aliquots zu je
        $150\mu l$ entnommen und bei $-80^{\circ}C$ gelagert.
   \end{itemize}
  \item \textbf{Speichelsammlung mittels SalivaGene Collection Module II für die DNA-Extraktion}
    \begin{itemize}
      \item Die Sammlung des Speichels erfolgt mittels der Saliva Gene
        Collector-Röhrchen nach Herstellerangaben. Es wird zwei
        Röhrchen befüllt. Das Röhrchen wird dicht verschlossen (bitte
        kontrollieren) und vorsichtig über Kopf für ca. 8
        Sek. geschüttelt.
      \item Das gesamte Röhrchen wird in der Biobank bei $-80^{\circ}C$ gelagert.
      \item \textbf{Achtung!} Die maximale Lagerzeit bis zur DNA-Extraktion
        sollte 12 Mon nicht überschreiten.
    \end{itemize}
\end{itemize}    



