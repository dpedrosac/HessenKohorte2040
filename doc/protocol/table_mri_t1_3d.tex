\begin{tabularx}{\linewidth}{@{} s | X @{}}
\caption{Details on T1-weighted \ac{MRI}-sequence}\\
%\small
\toprule
\multicolumn{2}{p{\dimexpr\linewidth-2\tabcolsep-2\arrayrulewidth}}{\textbf{T1-weighted, 3D volumetric \ac{MRI}-sequence during the \textsc{HessenKohorte}, e.g. \ac{MP-RAGE}, \ac{IR-FSPGR}}} \\
%\multicolumn{2}{l}{} \\
\midrule
Series description        & 3D T1-weighted                                                                                                              \\
Plane                     & Sagittal                                                                                                                    \\
Slice thickness (mm)      & 1.0 (slice thickness must remain consistent across timepoints)                                                              \\
Number of slices          & 192 (slice thicksness may be adjusted to 1.2 mm to cover brain if absolutely necessary. No adjustments of number of slices) \\
Voxel size (mm)           & 1.0 $\times$ 1.0 mm in plane resolution                                                                                     \\
Phase encode direction    & Anterior Posterior (AP)                                                                                                     \\
Matrix                    & 256 $\times$ 256 (the use of interpolation, zero-filling or a ZIP factor is not permitted)                                  \\
TR/TE/FA/other parameters & will be defined by Invicro according to the scanner                                                                         \\
\ac{FOV}                  & 256 mm (full FoV required, no rectangular FoV)                                                                              \\
Scan time                 & $\sim$ 7 min                                                                                                                \\
Further explanations      & The FOV must include the entire brain anatomy including the vertex, cerebellum and pons. Slices should be oblique sagittal, angled along the longitudinal fissure on both the axial and coronal localizers. To avoid artifacts, position the participant such that there is sufficient empty space around the head: approximately 1.5 cm of air or more above the top of the head, and leave 3 - 4 blank slices on either side of the head. Avoid nose ghosting.\\
\bottomrule
\multicolumn{2}{l}{\footnotesize{*protocol is identical to the one used by the \ac{PPMI}-study}}
\end{tabularx}
