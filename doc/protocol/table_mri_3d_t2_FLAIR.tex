\begin{tabularx}{\linewidth}{@{} s | X @{}}
\caption{Details on T2-weighted \ac{FLAIR} Sequence}\\
\toprule
\multicolumn{2}{p{\dimexpr\linewidth-2\tabcolsep-2\arrayrulewidth}}{\textbf{3D T2 \ac{FLAIR} Sequence}} \\
\midrule 
Series Description        & 3D T2 FLAIR                                                     \\
Plane                     & Sagittal                                                        \\
Slice thickness (mm)      & 1.0 -- 1.2 (slice thickness must remain consistent)             \\
Number of slices          & 192 (please adjust slice thickness up to \SI{1.2}{\milli\metre}
                            to cover brain, not the number of slices)                       \\
Voxel size (mm)           & 1.0 $\times$ 1.0 mm in plane resolution                         \\
Phase encode dir.         & Anterior-Posterior (AP)                                         \\
Matrix                    & 256 $\times$ 256 (the use of interpolation,
                            zero-filling or a ZIP factor is not permitted)                  \\
TR/TE/FA/other parameters & Will be defined by Invicro according to the scanner             \\
\ac{FOV}                  & \SI{256}{\milli\metre} (full \ac{FOV} required,
                            no rectangular \ac{FOV})                                        \\
Scan Time                 & $\sim$\SI{7}{\minute}                                           \\
Further explanations      &  The FOV must include the entire brain
                            anatomy including the vertex, cerebellum
                            and pons. To avoid artifacts, position the
                            participant such that there is sufficient
                            empty space around the head: approximately
                            \SI{1.5}{\centi\metre} of air or more
                            above the top of the head, and leave 1 --
                            2 blank slices on top of the head. Avoid
                            nose ghosting.                                                  \\
\end{tabularx}
