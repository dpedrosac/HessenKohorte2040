\begin{tabularx}{\textwidth}[!h]{X|X}
\caption{Inclusion and exclusion criteria for \ac{iPD}-patients to
  participate in the
  \textsc{HessenKohorte}}\label{tab:inclusionexclusionCriteriaPatients}\\ 
\toprule
\multicolumn{1}{c}{\textbf{inclusion criteria}} &
\multicolumn{1}{c}{\textbf{exclusion criteria}} \\ \toprule
\tabitem{Patients suffering from a clinical diagnosis of \ac{iPD}
  according to the recent clinical diagnostic criteria
  \cite{postuma2015mds}}

\tabitem{\ac{iPD}-stages of I - IV according to the Hoehn \& Yahr
  scale\cite{hoehn1967parkinsonism} (in the OFF stage, i.e., without
  medication)}

\tabitem{Patients with the ability to provide informed consent. In
  cases where participants lose this capacity at follow-up visits
  (e.g., due to dementia, etc.), participants will only be allowed to
  continue if legal representative provides informed consent to
  further participation on hisor her behalf. In this case, the legal
  representative will be provided with a separate consent form
  \ref{einfügen}}
&
\tabitem{Patients suffering from a clinical diagnosis of atypical
  Parkinson's syndrome in a first instance. Patients enrolled who were
  later characterized as atypical Parkinson syndroms will not be
  excluded.}

\tabitem{\ac{iPD}-stages of V according to the Hoehn \& Yahr scale
  \cite{hoehn1967parkinsonism} (in the OFF stage, i.e. without
  medication)}

\tabitem{The use of magnetic fields in the \ac{MRI} examination
  excludes the participation of persons who have electrical devices
  (e.g., cardiac pacemakers, medication pumps, etc.) or metal parts
  (e.g. screws after bone fracture) in or on their bodies.}

\tabitem{Women who are pregnant will not receive \ac{MRI} scans.} 

\tabitem{Subjects who do not want to be informed about possible
  incidental findings are also not allowed to participate in the
  imaging part of the study.} \\ \bottomrule
\end{tabularx}
