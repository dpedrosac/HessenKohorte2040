\begin{center}
  {\Huge SOP Blutbild -- Hessenkohorte 2040}
  
  Version 1.1 vom 17.07.2022 
\end{center}

\vspace*{1cm}

\begin{tabular}{@{}p{0.4\textwidth}l}
  erstellt von: Urs Kleinholdermann & am 17.07.2022 \\
  geprüft von: & am \\
  freigegeben von: & am \\
\end{tabular}

\vspace*{2cm}

\begin{center}
  {\huge Hessenkohorte 2040 Stud.Nr. 252 Institut für
    Laboratoriumsmedizin und Pathobiochemie}
\end{center}


\begin{enumerate}
  \item Parameter: Hämatologie: kleines Blutbild
  \item Abnahmeröhrchen: EDTA –Blut
  \item Formular immer mit der Angabe der Studiennummer 252 (für
    HK2040), ausgefüllt im Zentrallabor abgeben. Formulare unter
    ISF29.3
  \item Für Fragen steht der EDV – Beauftragten des Labors Herrn
    Patrick Junk zur Verfügung • Email: Patrick.Junk@uk-gm.de Tel.:
    66535
  \item Herrn Junk über die Anzahl der benötigten Laborzettel
    informieren und den Drucker (umrdr8335) nennen, auf dem die
    Laborergebnisse nach der Auswertung gesandt werden sollen
  \item Akkreditierungsurkunde und aktuelle Ringzertifikate der
    Parameter bei Fr. Pfeifer (leitende LMTA) im Zentrallabor
    anfordern: Email: \url{doris.pfeifer@uk-gm.de} Tel.: 64468
  \item Referenzwerte der Parameter im Intranet ausdrucken (Im
    Intranet unter \emph{Institut für Laboratoriumsmedizin} --
    \emph{Leistungsverzeichnis})
  \item Nach der Beendigung der Studie werden die Kosten mit dem Labor
    abgerechnet (Kostenberechnung : 3,5 Euro pro Blutbild – s. Mail
    Prof. Stief 02.06.21)
\end{enumerate}
