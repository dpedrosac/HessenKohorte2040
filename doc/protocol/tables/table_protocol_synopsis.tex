\begin{tabularx}{\textwidth}{p{3.5cm} | X}
\toprule
%\multicolumn{2}{c}{}\\
\multicolumn{2}{p{\dimexpr\linewidth-2\tabcolsep-2\arrayrulewidth}}
{\textbf{Longitudinal digital observation of the holistic quality of the life of patients with \acl{iPD} and their caregivers: a prospective observational cohort study
}}
\\ \toprule

\textbf{Study objectives} & 
The aim of this study is to monitor the \ac{QoL} of \num[round-precision = 0, round-mode = places]{1000} patients suffering from \ac{iPD} and their relatives over 20 years and to relate this to objectifiable changes in metabolism, but also to structural imaging changes during this time.
\\ \midrule

\textbf{Study design} &
Prospective single-center observational cohort study
\\ \midrule

\textbf{Planned Number of Subjects} &
\num[round-precision = 0, round-mode = places]{1000} 
\\ \midrule

\textbf{Primary Endpoint} &
Quality of life of the index patient after an observation of up to 20 years
\\ \midrule

\textbf{Secondary Endpoints} & 
\begin{tabitemize}
\item Quality of life of patients' relatives after up to 20 years of follow-up
\item Changes in motor symptoms of the index patient at up to 20 years
\item Development of non-motor symptoms over up to 20 years
\item Changes in functional imaging over the follow-up period
\end{tabitemize}
\\ \midrule

\textbf{Enrollment of participants} & Patients suffering from \ac{iPD} can be enrolled together with their relatives at any point in time.
\\ \midrule

\textbf{Study visits schedule} & 
\begin{tabitemize}
\item Screening 
\item Baseline Visit 
\item Semiannual visit 
\item Annual visit 
\item Unscheduled visits 
\end{tabitemize}
\\ \midrule 

\textbf{Study duration} &
The study will be considered complete after all subjects complete their visit in the year 2043. Hence, the total study duration is estimated to be at most 20 years.
\\ \midrule

\textbf{Inclusion criteria \ac{iPD}-patients} &
\begin{tabitemize}
\item Patients suffering from a clinical diagnosis of \acs{iPD}
  according to the recent clinical diagnostic criteria
  \cite{postuma2015mds}
\item \ac{iPD}-stages of \RNum{1} -- \RNum{4} according to the Hoehn
  \& Yahr scale (in the OFF state, i.e., without medication)
  \cite{hoehn1967parkinsonism}
\item Patients aged between between 30 and 100 years
\item Patients with the ability to provide informed consent. In
  cases where participants lose their capacity to consent at follow-up
  visits (e.g., due to dementia, etc.), this participant will only be
  allowed to continue if a legal representative (proxy, guardian)
  provides informed consent to further participation on behalf of the
  participant. In this case, the legal representatives will be
  provided with a separate consent form.
\end{tabitemize}
\\ \midrule

\textbf{Exclusion criteria \ac{iPD}-patients} &
\begin{tabitemize}
\item Patients suffering from a clinical diagnosis of atypical 
  Parkinson's syndrome in a first instance. Patients enrolled who
  were later characterized as atypical Parkinson syndroms will not be
  excluded.
\item \ac{iPD}-stages of \RNum{5} according to the Hoehn \& Yahr scale
  (in the OFF stage, i.e. without medication) \cite{hoehn1967parkinsonism}
\item The use of magnetic fields in the MRI examination excludes
  the participation of persons who have electrical devices
  (e.g., cardiac pacemakers, medication pumps, etc.) or metal parts
  (e.g., screws after bone fracture) in or on their bodies.
\item Women who are pregnant will not receive \ac{MRI}.
\item Subjects who do not want to be informed about possible
  incidental findings are also not allowed to participate in the
  imaging part of the study.
\end{tabitemize}
\\ \midrule

\textbf{Inclusion criteria \ac{iPD}-patients' relatives} &
\begin{tabitemize}
\item Relatives of patients included in the study according to the abovementioned criteria
\item Subjects with the ability to give informed consent
\end{tabitemize}
\\ \midrule

\textbf{Exclusion criteria \ac{iPD}-patients' relatives} &
Relatives who are unable to give informed consent cannot participate in the study
\\ \midrule

\textbf{Statistical methods} &
Due to the large number of possible primary endpoints, a variety of analyses will be conducted to examine changes and variability over the 20 years. Analyses will include logistic, linear and non-linear models to assess these data. Primary interest will focus on established and long-term measures of quality of life (such as the \ac{PDQ39}\cite{jenkinson1997pdq39}) but also of clinical severity such as the \ac{MDS-UPDRS}\cite{goetz2007updrs}. For secondary endpoints, clinical, imaging and biological validation studies will be performed on promising biological markers in study subsets using stored collected samples. These analyses may vary considerably depending on the type of marker and available data.\\ \midrule

\textbf{Statistical test method} & 
Regression models will be performed to determine the predictors of decline in \ac{QoL} during the course of \ac{iPD} in relation to the primary endpoint and the planned number of participants. In addition, studies correlating \ac{QoL} with the identified markers and imaging results should be performed.\\ \midrule

\textbf{Sample Size Parameters} & 
This is a longitudinal cohort study where the number of participants is determined by the number of resources available. The number of participants is therefore a combination of the expected number of patients in the district and the resources available to the \UKM.
\\ \bottomrule
\end{tabularx}
