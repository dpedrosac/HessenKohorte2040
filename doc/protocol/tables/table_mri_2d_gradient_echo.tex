\begin{tabularx}{\linewidth}{@{} s | X @{}}
\caption{Details on T2-weighted \ac{MRI}-sequence}\\
%\small
%\setstretch{1}
\toprule
\multicolumn{2}{p{\dimexpr\linewidth-2\tabcolsep-2\arrayrulewidth}}%
{\textbf{2D-Gradient-echo T2*-weighted \ac{EPI} (e.g., ep2d\_BOLD)}} \\
\midrule                                                                                                                                                                            
Series description     & rsfMRI\_RL \\
Plane                  & Axial oblique plane, parallel to AC--PC line \\
Slice thickness        & \SI{3.5}{\milli\metre} with no gap \\
Number of slices       & $\sim$40 \\
Phase encode direction & R $\gg$ L\\
Matrix                 & 64 $\times$ 64 \\
\ac{FoV}               & 224 $\times$ 224mm \\
Repetition time        & 2500ms \\
Echo time              & 30ms\\
Flip angle             & 80deg\\
Slice order            & Interleaved \\
Number of measurements & 240 (10min total scan time) \\
In-plane acceleration  & GRAPPA or SENSE (factor of 2) \\
Scan time              & $\sim$10min\\
Further explanations   & Participants should be instructed to keep their eyes open throughout     
                         the scan and to focus on a point on the mirror or scanner. Immediately   
                         after the scan, participants should be checked to ensure that they       
                         have kept their eyes open and have not fallen asleep. No audio or        
                         video presentations should be made during the scan. Axial resting        
                         state fMRI slices should be positioned along the AC--PC plane, with      
                         one slice above the vertex and the remaining slices maximally covering   
                         the brain and as much of the cerebellum as possible. The slices should   
                         be centred in the axial plane to avoid anterior/posterior aliasing and   
                         \ac{TR}, \ac{TE} should not be altered. \\ 
\bottomrule
\end{tabularx}
