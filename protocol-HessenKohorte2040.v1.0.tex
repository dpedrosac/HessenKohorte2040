%%%%%%%%%%%% 			Start with the document  			%%%%%%%%%%%%%
\documentclass[
	a4paper, 
	11.5pt,
	headings=small, 
	twoside, 
	titlepage=firstiscover, 
 	pagesize=auto,
  	version=last,
	open=any,
	BCOR=14mm,
  	chapterprefix=false]{scrbook}
\setkomafont{sectioning}{\rmfamily}
\usepackage{bookman}
\usepackage{gitfile-info}
\KOMAoptions{DIV=14}

%%%%%%%%%%%%				Präambel				%%%%%%%%%%%%%
\usepackage{ifluatex,ifxetex}
\ifluatex\else\ifxetex\else
  \usepackage[T1]{fontenc}
  \usepackage[utf8]{inputenc}
\fi\fi
\usepackage[ngerman]{babel}
\addto{\captionsngerman}{
  \renewcommand*\contentsname{Inhalt}}
\usepackage{expl3}
\csname sys_if_shell_unrestricted:T\endcsname{\usepackage{gitver}} 
\usepackage{pdflscape}

%\usepackage[bindingoffset=2mm]{geometry}% Bindeverlust von 8mm einbeziehen
\usepackage{siunitx}
\sisetup{per-mode = symbol, round-mode=places, detect-all}

% \usepackage{ebgaramond}
\usepackage{setspace}
\setstretch{1.5}
\usepackage{nameref}
\usepackage{lmodern}
\usepackage{multirow}
\usepackage{booktabs,tabularx,dcolumn, makecell}
\newcommand{\tabitem}{~~\llap{\textbullet}~~}
\usepackage[printonlyused, smaller]{acronym}
\usepackage{graphicx}
\usepackage{float}
\graphicspath{ {media/} }

\usepackage[
	font=footnotesize, 
	format=plain, % Die Beschriftung als Absatz.
 	indention=1.5cm, % Einzug der Beschriftung ist 1cm.
 	textformat=simple, % Der Text soll nicht verändert werden.
 	labelfont=footnotesize, % Der Bezeichner soll groß und fett geschrieben werden.
 	textfont=it % Der Text soll kursiv gesetzt werden.
]{caption}

\usepackage[headsepline,automark,markcase=noupper]{scrlayer-scrpage} %Titel in Kopfzeile
\renewcommand*{\sectionmarkformat}{\thesection. \ }

\usepackage{xcolor}
\usepackage[bottom]{footmisc}
\usepackage{microtype}
\usepackage[round]{natbib}
\bibliographystyle{abbrvnat}
\usepackage{multicol}
\usepackage{pdfpages}
\usepackage[export]{adjustbox}

\usepackage{anyfontsize}
\setkomafont{chapter}{\normalfont\Large}
\renewcommand*{\chapterheadstartvskip}{\vspace*{.80\baselineskip}}
\renewcommand*{\chapterheadendvskip}{\vspace*{.5\baselineskip}}
\renewcommand*{\chapterformat}{%
  {\fontsize{25}{30}\scshape\chapappifchapterprefix{}}%
  \fontsize{64}{30}\selectfont\rlap{\thechapter\autodot}%
}
\renewcommand*{\raggedchapter}{\raggedleft}

\renewcommand*\othersectionlevelsformat[3]{%
  \llap{#3\autodot\enskip}%
}

\setkomafont{section}{\Large\rmfamily}
\renewcommand{\thesubsubsection}{\alph{subsubsection}}
\setcounter{secnumdepth}{4}

\usepackage{blindtext}
\definecolor{headlinie}{RGB}{84,84,84}
\setcapindent{5pt}
\usepackage[headsepline,automark]{scrlayer-scrpage}
\addtokomafont{headsepline}{\color{headlinie}}
\KOMAoptions{headsepline=.1pt}
	
\usepackage{lipsum}  
\usepackage{hyperref}

 %%%%%%%%%%%%				Metadaten  				%%%%%%%%%%%%
\hypersetup{
	pdftitle={Protocol - HessenKohorte, Neurology Department, Philipps-University Marburg},
	pdfsubject={Habilitationsschrift},
	pdfauthor={PD Dr. med. David J. Pedrosa},
	pdfkeywords={Parkinson's disease, cohort study, hassia, quality-of-life}
}

\newcolumntype{d}[1]{D..{#1}}
\newcolumntype{C}{>{\centering\arraybackslash}X}
\newcommand\mc[1]{\multicolumn{1}{C}{#1}} % shortcut macro
\def\sym#1{\ifmmode^{#1}\else\(^{#1}\)\fi}
\newcommand{\UKGM}[1]{University Hospital of Gießen and Marburg}

\begin{document}
%% Title Page (missing)

 %%%%%%%%%%%%  				Abkürzungsverzeichnis  				%%%%%%%%%%%%

\chapter*{Abbreviations}
\phantomsection
\thispagestyle{plain}
\begin{multicols}{2}
\begin{acronym}
\acro{BDI}[BDI]{Beck's Depression Inventory}
\acro{CBI}[CBI]{Copenhagen Burnout Inventory}
\acro{CISS}[CISS]{Coping Inventory for Stressful Situations}

\acro{CHAPO}[CHAPO]{Challenges and Opportunities model}
\acro{CHAPO-PD}[CHAPO]{Challenges and Opportunities for Parkinson's Disease Patients}
\acro{CRF}[CRF]{case report form}
\acro{DBS}[DBS]{Deep Brain Stimulation}
\acro{EPI}[EPI]{Echo planar imaging}
\acro{EC}[EC]{Ethics Committee}
\acro{EEG}[EEG]{Electroencephalography}
\acro{GPe}[GPe]{\textit{Globus pallidus, pars externa}}
\acro{ICF}[ICF]{Informed consent form}

\acro{iPS}[iPS]{idiopathic Parkinson's syndrome}
\acro{GPi}[GPi]{\textit{Globus pallidus, pars interna}}
\acro{MFI-20}[MFI-20]{Multidimensional Fatigue Inventory}

\acro{MDS-UPDRS}[MDS-UPDRS]{Movement Disorder Society Unified Parkinson's Disease Rating Scale}
\acro{MoCa}[MoCa]{Montréal Cognitive Assessment}
\acro{MNI}[MNI]{Montréal Neurological Institute}
\acro{MRI}[MRI]{Magnetic resonance imaging}
\acro{NMSQ}[NMSQ]{Non-Motor Symptom Questionnaire}

\acro{PD}[PD]{Parkinson's disease}
\acro{PHQ}[PHQ]{Patient Health Questionnaire}
\acro{PDCB}[PDCB]{Parkinson’s disease caregiver burden Questionnaire PDCB}
\acro{PPMI}[PPMI]{Parkinson's Progressive Marker Initiative}
\acro{PSS}[PSS]{Perceived Stress Scale}

\acro{QoL}[QoL]{Quality-of-Life}
\acro{SNc}[SNc]{\textit{Substantia nigra, pars compacta}}
\acro{STN}[STN]{\textit{Nucleus subthalamicus}}
\acro{TE}[TE]{\textit{echo time}}
\acro{TI}[TI]{\textit{inversion time}}
\acro{TR}[TR]{\textit{repetition time}}

\acro{WHOQoL}[WHOQoL]{World Health Organization Quality of Life}
\acro{ZBI-22}[ZBI-22]{Zarit Burden Interview}
\end{acronym}
\end{multicols}

\chapter{Preamble}
\section{Contact information}
% \caption{A \texttt{tabularx} solution}
\begin{tabularx}{0.85\textwidth}{@{}lp{8cm} *{1}{l} @{}}
\toprule
\textbf{Role} & \textbf{Contact} \\
\cmidrule{1-2}
Clinical contact 	& PD Dr. David Pedrosa \\
			& Neurology Department, \UKGM (Marburg site)\\
			& Baldingerstr.\\
			& 35043 Marburg\\
			& \href{mailto:david.pedrosa@staff.uni-marburg.de}{david.pedrosa@staff.uni-marburg.de}\\
\bottomrule
Responsible for \ac{MRI}-acquisition & Marina Ruppert, M.Sc.\\
			& Neurology Department, \UKGM (Marburg site)\\
			& Baldingerstr.\\
			& 35043 Marburg\\
			& \href{mailto:marina.ruppert@uni-marburg.de}{marina.ruppert@uni-marburg.de}\\
\bottomrule
Data protection officer & \\
			& XXX\\
			& XXX\\
			& XXX\\
			& XXX\\

\end{tabularx}
\newpage

\section{Version}
\begin{tabularx}{1\textwidth}{@{}C *{5}{C}@{}}
\toprule
\textbf{Revision Version} & \textbf{Protocol Date} & \textbf{Template number and version} & \textbf{Protocol Section modified} & \textbf{Summary of Changes} & \textbf{Justification for Modification}\\
\midrule
A & Mai 19th, 2022 &  & - & Initial version & Initial version \\
\cmidrule{2-6}

\end{tabularx}


\chapter{Protocol}
\section{Introduction}
\subsection{Background}
Parkinson's syndrome represent chronic neurodegenerative conditions manifesting with with both motor and non-motor symptoms. The symptoms have a major psychosocial impact and lead to considerable losses in the quality of life of patients and a high burden on (informal) caregivers. So far, various instruments have been developed to assess quality of life, some of them specific to \ac{PD}. However, none of the models takes into account the positive aspects of well-being, as well as the personal attitiude (e.g., optimism)  and social resilience factors (e.g., social support, social integration, etc.). In the context of the project, an investigation of the quality of life of \ac{PD}-patients will be carried out. For this purpose, the quality of life of patients will be examined primarily longitudinally with established and validated questionnaires over a very long period of time. In a further step, it will be examined whether holistic observations of the quality of life can also provide meaningful statements. The envisaged instrument for recording this, the so-called \ac{CHAPO} model, an approach initially developed for evaluating the quality of life of the very old (NRW80+ (Quelle?)). By adapting it to aspects of quality of life specific to Parkinson's disease, as the \acs{CHAPO-PD} model, it will find application in this cohort study. Thus, the aim of the study is the assessment of quality of life using digital solutions with classical parameters for the evaluation of \acl{QoL} such as symptoms of the diseases and limitations in everyday life. The aim is to establish a digital \ac{QoL} monitoring system with the goal of assessing Parkinson's patients in a standardized way throughout the course of their disease and to evaluate their quality of life.
We would like to relate the data obtained in this way to the annual follow-up cranial MRI and the biomedical markers obtained from stool, urine, saliva, hair and blood samples. This would conceivably identify imaging or biomedical markers with predictive value for quality of life change.

In addition, the study will include a follow-up assessment of necessary support services to improve understanding of the needs of family members of \ac{PD}-patients. This should make it possible to develop a demand-oriented support offer according to the different phases of the disease. The stress experience, changes in sleep behavior and the losses in quality of life over the observation period are also included in the analysis in order to detect a surrogate for adequate support.


\subsection{Geographic context}
Approximately \SI{400000} people are diagnosed with \ac{PD} in Gernany with an incidence of \SI{84.1} per \SI{100000} patients annually [12]. The \UKGM treates up to 1500 patients. In order to understand the special recruitment peculiarities of the University Hospital of Marburg, the knowledge about the location of Marburg in Germany is essential, because Marburg is situated in the countryside in the western center of Germany with 77,129 inhabitants [18], although it is a university city and district town in Hesse. Due to its location about 77 kilometers of direct distance between the metropolitan areas of Frankfurt am Main and Kassel, the importance of the University Hospital of Marburg in the medical care of the district 70 km around Marburg is essential. To ensure patient’s access to care in the district of Marburg, the Parkinson Network Alliance Marburg (PANAMA) was created in 2016 by the Department of Neurology at the Marburg University Hospital. Within this care network different stakeholders work together to facilitate the integration of care services and improve outcomes for patients. 

To represent diversity of the real-life population of PD patients in the district of the Marburg University Hospital, and to guarantee a balanced study cohort, it is intended to include all patients with parkinsonism in the district. Therefore, successfully approved recruitment strategies used in past clinical trials needed to be adopted and refined. First, the patients are directly asked to participate within their appointments in the outpatient clinic or the hospital stay in the Department of Neurology in the University Hospital Marburg. Second, the PANAMA healthcare network will bring the study to the attention of its members, so that the associated healthcare providers will further promote the study. Third, media appearance of the University Hospital will give detailed information (https://www.uni-marburg.de). Additionally, the associated caregivers are requested to participate. 

\section{Protocol synopsis}
\begin{tabularx}{1\textwidth}{m{3.5cm} | X}
\toprule
%\multicolumn{2}{c}{}\\
\multicolumn{2}{p{\dimexpr\linewidth-2\tabcolsep-2\arrayrulewidth}|}{\textbf{Longitudinal digital observation of the holistic quality of the life of patients with \ac{PD} and their caregivers: a prospective observational cohort study}} \\
\toprule
\textbf{Study objectives} 			& This study aims at observing quality of life of 1000 patients suffering from \ac{PD} and their relatives over the course of 20 years and relating this to a objetifiable changes in the metabolism but also to structural imaging changes in this period \\
\midrule
\textbf{Study design} 				& Prospective single-center cohort study\\
\midrule
\textbf{Planned Number of Subjects} & 1000 \\
\midrule
\textbf{Planned Number of Subjects} & 1000 \\
\midrule
\textbf{Primary Endpoint} 			& \\
\midrule
\textbf{Secondary Endpoints} 		& \\
\midrule
\textbf{Enrollment of participants} 	& Patients suffering from \ac{PD} may be enrolled together with their relatives at any point in time\\
\midrule
\textbf{Study visits schedule} 		& \tabitem{Screening}\\
							& \tabitem{Baseline Visit}\\
							& \tabitem{Yearly follow-up}\\
							& \tabitem{Visit at year 2042 will be teh End of Study Visit}\\
\midrule
\textbf{Study Duration} 			& The study will be considered complete after all subjects complete their visit in the year 2042. The total study duration is estimated to be at most 20 years. \\
\midrule
\textbf{Inclusion criteria \ac{PD}-patients}& \tabitem{Patients suffering from a clinical diagnosis of idiopathic Parkinson's syndrome according to the recent clinical diagnostic criteria (Quelle Postuma)} \\
							& \tabitem{\ac{iPS}-stages of I - IV according to the Hoehn \& Yahr scale (without medication, i.e. in the OFF stage) (Quelle Hoehn und Yahr 1967).} \\
& \tabitem{Patients aged between between 30 and 100 years} \\
							& \tabitem{Patients with the ability to provide informed consent. In cases where participants lose their capacity to consent at follow-up visits (e.g., due to dementia, etc.), this participant will only be allowed to continue if a legal representative (proxy, guardian) provides informed consent to further participation on behalf of the participant. In this case, the legal representatives will be provided with a separate consent form.} \\
							& \tabitem{Patients with a good knowledge of German} \\
\midrule
\textbf{Exclusion criteria \ac{PD}-patients}& \tabitem{Patients suffering from a clinical diagnosis of atypic Parkinson's syndrome in a first instance. Patient's enrolled who were later characterized as atypical Parkinson syndroms will not be excluded.}\\
							& \tabitem{\ac{iPS}-stages of V according to the Hoehn \& Yahr scale (without medication, i.e. in the OFF stage) (Quelle Hoehn und Yahr 1967)}\\
							& \tabitem{The use of magnetic fields in the MRI examination excludes the participation of persons who have electrical devices (e.g. cardiac pacemakers, medication pumps, etc.) or metal parts (e.g. screws after bone fracture) in or on their bodies.}\\
							& \tabitem{Women who are pregnant will not receive MR imaging.}\\
							& \tabitem{Subjects who do not want to be informed about possible incidental findings are also not allowed to participate in the imaging part of the study.}\\
\midrule
\textbf{Inclusion criteria \ac{PD}-patients' relatives}& \tabitem{Patients suffering from a clinical diagnosis of idiopathic Parkinson's syndrome according to the recent clinical diagnostic criteria (Quelle Postuma)} \\
							& \tabitem{\ac{iPS}-stages of I - IV according to the Hoehn \& Yahr scale (without medication, i.e. in the OFF stage) (Quelle Hoehn und Yahr 1967).} \\
& \tabitem{Patients aged between between 30 and 100 years} \\
							& \tabitem{Patients with the ability to provide informed consent. In cases where participants lose their capacity to consent at follow-up visits (e.g., due to dementia, etc.), this participant will only be allowed to continue if a legal representative (proxy, guardian) provides informed consent to further participation on behalf of the participant. In this case, the legal representatives will be provided with a separate consent form.} \\
							& \tabitem{Patients with a good knowledge of German} \\
\midrule
\textbf{Exclusion criteria \ac{PD}-patients' relatives}& \tabitem{Patients suffering from a clinical diagnosis of atypic Parkinson's syndrome in a first instance. Patient's enrolled who were later characterized as atypical Parkinson syndroms will not be excluded.}\\
							& \tabitem{\ac{iPS}-stages of V according to the Hoehn \& Yahr scale (without medication, i.e. in the OFF stage) (Quelle Hoehn und Yahr 1967)}\\
							& \tabitem{The use of magnetic fields in the MRI examination excludes the participation of persons who have electrical devices (e.g. cardiac pacemakers, medication pumps, etc.) or metal parts (e.g. screws after bone fracture) in or on their bodies.}\\
							& \tabitem{Women who are pregnant will not receive MR imaging.}\\
							& \tabitem{Subjects who do not want to be informed about possible incidental findings are also not allowed to participate in the imaging part of the study.}\\
\midrule


\textbf{Statistical methods} 		& \\
\textbf{Primary Statistical Hypothesis}& \\
\textbf{Statistical test method} 		& \\
\textbf{Sample Size Parameters} 		& \\
\bottomrule
\end{tabularx}
\newpage



\section{Study objectives and Endpoints}
\section{Study design}
The \textsc{HessenKohorte} Study is a prospective 20-year cohort study. The \textsc{HessenKohorte} study aims to recruit patients of all genders suffering from clinically probable \ac{PD} (n = 1000) along with their relatives in the German region of Hessen. All patients will be recruited from the treated patients in an in- and outpatient setting between 2022 and 2042. 

\subsection{Scale and duration}
The study will accompany up to 1000 patients over as much as 20 years in order to enable a good insight into the course of the individual patients and their relatives.
\subsection{Justification for study design}
This study is a single-center prospective and longitudinal cohort study to monitor \ac{PD}-patients development over the course and particularly to assess their quality of life in a holistic approach. The comparatively large number of subjects will enable a better insight into \ac{PD} with its multifacetted phenotypes

\section{Subject selection}
\label{sec:study_selection}
\subsection{Study population and Eligibility}
\label{sec:study_population}
Study candidates will be drawn from the patients treated in the Neurology Department of the \UKGM (Marburg site) as either in- or outpatients. Moreover, patients suffering from \ac{PD} may submit a request for participation in the study. The inclusion and exclusion criteria (cf. Section \ref{sec:study_population}) are checked by one of the study physicians, who are responsible for the final decision. Advertising for the study can be found in the form of a flyer, which is available in the Department of Neurology, but also in the form of an Internet, where the project is presented.

% Create a table with inclusion criteria with 2 columns, {num_IC} rows  with bulletpoints and a caption to be referred to
\subsection{Inclusion criteria \ac{PD}-patients}
\label{sec:inclusion_criteriaIPS}
Subjects who meet all of the following criteria (cf. Table \ref{}) may be given consideration for inclusion in this cohort study, provided no exclusion criteria (cf. Section \ref{sec:exclusion_criteriaIPS}) are met.

\begin{itemize}
\item Patients suffering from a clinical diagnosis of idiopathic Parkinson's syndrome according to the recent clinical diagnostic criteria (Quelle Postuma).
\item \ac{iPS}-stages of I - IV according to the Hoehn \& Yahr scale (without medication, i.e. in the OFF stage) (Quelle Hoehn und Yahr 1967).
\item Patients aged between between 30 and 100 years
\item Patients with the ability to provide informed consent. In cases where participants lose their capacity to consent at follow-up visits (e.g., due to dementia, etc.), this participant will only be allowed to continue if a legal representative (proxy, guardian) provides informed consent to further participation on behalf of the participant. In this case, the legal representatives will be provided with a separate consent form.   
\item Patients with a good knowledge of German
\end{itemize}

% Create a table with exclusion criteria with 2 columns, {num_IC} rows  with bulletpoints and a caption to be referred to
\subsection{Exclusion criteria \ac{PD}-patients}
\label{sec:exclusion_criteriaIPS}
Subjects who meet any one of the following criteria (cf. Table \ref{}) cannot be included or will be excluded from this cohort study.
\begin{itemize}
\item Patients suffering from a clinical diagnosis of atypic Parkinson's syndrome in a first instance. Patient's enrolled who were later characterized as atypical Parkinson syndroms will not be excluded.
\item \ac{iPS}-stages of V according to the Hoehn \& Yahr scale (without medication, i.e. in the OFF stage) (Quelle Hoehn und Yahr 1967).
\item The use of magnetic fields in the MRI examination excludes the participation of persons who have electrical devices (e.g. cardiac pacemakers, medication pumps, etc.) or metal parts (e.g. screws after bone fracture) in or on their bodies. 
\item Women who are pregnant will not receive \ac{MRI} scans.
\item Subjects who do not want to be informed about possible incidental findings are also not allowed to participate in the imaging part of the study.
\end{itemize}

% Create a table with inclusion criteria with 2 columns, {num_IC} rows  with bulletpoints and a caption to be referred to
\subsection{Inclusion criteria \ac{PD}-patients' relatives}
\label{sec:inclusion_criteriaREL}
Subjects who meet all of the following criteria (cf. Table \ref{}) may be given consideration for inclusion in this cohort study, provided no exclusion criteria (cf. Section \ref{sec:exclusion_criteriaREL}) are met.

\begin{itemize}
\item Patients suffering from a clinical diagnosis of idiopathic Parkinson's syndrome according to the recent clinical diagnostic criteria (Quelle Postuma).
\item \ac{iPS}-stages of I - IV according to the Hoehn \& Yahr scale (without medication, i.e. in the OFF stage) (Quelle Hoehn und Yahr 1967).
\item Patients aged between between 30 and 100 years
\item Patients with the ability to provide informed consent. In cases where participants lose their capacity to consent at follow-up visits (e.g., due to dementia, etc.), this participant will only be allowed to continue if a legal representative (proxy, guardian) provides informed consent to further participation on behalf of the participant. In this case, the legal representatives will be provided with a separate consent form.   
\item Patients with a good knowledge of German
\end{itemize}

% Create a table with exclusion criteria with 2 columns, {num_IC} rows  with bulletpoints and a caption to be referred to
\subsection{Exclusion criteria \ac{PD}-patients' relatives}
\label{sec:exclusion_criteriaREL}
Subjects who meet any one of the following criteria (cf. Table \ref{}) cannot be included or will be excluded from this cohort study.
\begin{itemize}
\item Patients suffering from a clinical diagnosis of atypic Parkinson's syndrome in a first instance. Patient's enrolled who were later characterized as atypical Parkinson syndroms will not be excluded.
\item \ac{iPS}-stages of V according to the Hoehn \& Yahr scale (without medication, i.e. in the OFF stage) (Quelle Hoehn und Yahr 1967).
\item The use of magnetic fields in the MRI examination excludes the participation of persons who have electrical devices (e.g. cardiac pacemakers, medication pumps, etc.) or metal parts (e.g. screws after bone fracture) in or on their bodies. 
\item Women who are pregnant will not receive \ac{MRI} scans.
\item Subjects who do not want to be informed about possible incidental findings are also not allowed to participate in the imaging part of the study.
\end{itemize}


\section{Subject accountability}

\subsection{Point of enrollment}
A subject will be considered enrolled at the time of the study-specific informed consent form (ICF) execution. No study-related procedures or assessments can take place until the ICF is signed.

\subsection{Withdrawal}
All subjects enrolled in the \textsc{HessenKohorte} (including those withdrawn from the clinical study) shall be accounted for and documented. If a subject withdraws from the clinical investigation, the reasons shall be reported.

Reasons for withdrawal include but are not limited to:
\begin{itemize}
\item subject or relative choice to withdraw consent,
\item lost to follow-up,
\item pregnancy*,
\item implantation of electrical devices or metal parts in or on the body *
\end{itemize}

* Only the MR-imaging will be discontinued during pregnancy or from the moment of an implantation onwards.

Subjects may oif course withdraw at any time, with or without reason, and without prejudice to further treatment. All applicable case report forms \ac{CRF} up to the point of subject withdrawal and an ``End of Study'' form must be completed. Any subject deemed ``lost to follow-up'' should have a minimum of three documented attempts to contact him/her prior to completion of the ``End of Study'' form. Additional study data may no longer be collected after the point at which a subject has been withdrawn from the study or withdraws consent, for whatever reason. Data collected up to the point of subject withdrawal may be used. Subjects withdrawn after completing the implant procedure will not be replaced 
%% Zu Klären End of Study Form? Sollen ausgeschiedene ersetzt werden?.

\subsection{Lost to follow-up}
% Was passiert im Falle einer Scheidung/des Tods eines Angehörigen bzw. wenn ein anderer Angehöriger teilnehmen möchte?
\subsection{Subject status and classification}
A subject will be considered enrolled in this study at the time of the study-specific \ac{ICF} execution.

\subsection{Enrolment control}
The overall enrollment in the study will be capped at 1000 participants.

\subsection{End-of-study definition}
The study is considered complete when 20 years from the first enrolment are over.

\section{Study methods}
\subsection{Data collection}
The data collection schedule is shown in Table \ref{}
\newpage
\begin{landscape}
\begin{table}
\caption{Data Collection Schedule for \ac{PD}-patients enrolled in the \textsc{HessenKohorte}}
\begin{tabularx}{1\textwidth}{@{}X *{6}{C}@{}}
\toprule
\textbf{Visit} 				& \textbf{Screening} 	& \textbf{Baseline visit} 	& \textbf{Half-year visit} 	& \textbf{Year 1,2,3,4, ..., 20 Visit} 	& \textbf{Unscheduled Visit} 	\\
\cmidrule{2-6}
Informed Consent Process 	& X 					&  						& 						& 								& 							\\
Eligibility Criteria			& \multicolumn{2}{c}{X}							& 						& 								& 							\\
Subject Demographics 		& \multicolumn{2}{c}{X}							& X 						& X 								& 							\\
\ac{MDS-UPDRS} 			& \multicolumn{2}{c}{X}							&  						& X 								& X*							\\
\ac{NMSQ}				& \multicolumn{2}{c}{X}							&						& X								&							\\
\ac{CHAPO-PD}			& \multicolumn{2}{c}{X}							&						& X								&							\\
Hair sample				& \multicolumn{2}{c}{X}							&						& X								&							\\
Saliva sample				& \multicolumn{2}{c}{X}							&						& X								&							\\
Blood  sample			& \multicolumn{2}{c}{X}							&						& X								&							\\
Stool sample				& \multicolumn{2}{c}{X}							&						& X								&							\\
\bottomrule
\multicolumn{6}{l}{\footnotesize{*may be ascertained and entered into database}}
\end{tabularx}
\end{table}
\newpage

\begin{table}
\caption{Data Collection Schedule for patients' relatives enrolled in the \textsc{HessenKohorte}}
\begin{tabularx}{1\textwidth}{@{}X *{6}{C}@{}}
\toprule
\textbf{Visit} 				& \textbf{Screening} 	& \textbf{Baseline visit} 	& \textbf{Half-year visit} 	& \textbf{Year 1,2,3,4,5, ..., 20 Visit} 	& \textbf{Unscheduled Visit} 	\\
\cmidrule{2-6}
Informed Consent Process 	& X 					&  						& 						& 								& 							\\
Eligibility Criteria			& \multicolumn{2}{c}{X}							& 						& 								& 							\\
Subject Demographics 		& \multicolumn{2}{c}{X}							& X 						& X 								& 							\\

Hair sample				& \multicolumn{2}{c}{X}							&						& X								&							\\
Saliva sample				& \multicolumn{2}{c}{X}							&						& X								&							\\
Blood  sample			& \multicolumn{2}{c}{X}							&						& X								&							\\
Stool sample				& \multicolumn{2}{c}{X}							&						& X								&							\\
\bottomrule
\multicolumn{6}{l}{\footnotesize{*may be ascertained and entered into database}}
\end{tabularx}
\end{table}
\end{landscape}

\subsection{Candidate Screening}
\label{subsec:screening}
Subjects will be screened for participation in the study based on study Inclusion and exclusion criteria as listed in Section \ref{sec:study_selection}. Subjects who have provided informed consent and who have been determined to not meet all eligibility requirements will be withdrawn.

\subsection{Informed consent}
Written informed consent must be obtained from potential study candidates and enrollment is only valid, after subjects sign and date the \ac{ICF}.
\begin{itemize}
\item Subjects will be asked to sign the \ac{ICF} before study-specific tests or procedures are performed;
\item The idea of the study must be explained, and subjects must be given the time and opportunity to ask questions and have those questions answered to their satisfaction.
\item The \ac{ICF} is study specific and has been approved by the\ac{EC}.
\item Written informed consent must be recorded appropriately by means of the subject’s dated signature.
\end{itemize}

\subsection{Questionnaires}
\label{subsec:questionnaires}
\subsubsection{\acl{CHAPO-PD}}

\subsubsection{\acl{MDS-UPDRS}}
Der MDS-UPDRS ist ein Assessment, das zur Beurteilung verschiedener Aspekte der Parkinson-Krankheit dient, einschließlich nicht-motorischer und motorischer Symptome. Der MDS-UPDRS besteht aus vier Teilen:
Teil I: Erfahrungen des täglichen Lebens (nicht-motorische Aspekte). Umfasst 13 Items.
IA: Dieser Teil bezieht sich auf verschiedene Verhaltensauffälligkeiten des Patienten, die durch den Untersucher evaluiert werden.
IB: Dieser Teil soll vom Patienten selbst ausgefüllt werden, ggf. mit Unterstützung einer Betreuungsperson, jedoch unabhängig vom Untersucher.
Teil II: Erfahrungen des täglichen Lebens (motorische Aspekte): Dieser Teil ist ebenfalls ein Selbstbefragungsbogen, der vom Patienten ausgefüllt werden soll, ggf. mit Unterstützung einer Betreuungsperson, jedoch unabhängig vom Untersucher. Umfasst 13 Items.
Teil III: Motorische Untersuchung: Die Instruktionen werden dem Patienten vom Untersucher vorgelesen oder direkt demonstriert-. Dieser Teil wird durch den Untersucher ausgefüllt. Umfasst 18 Items.
Teil IV: Motorische Komplikationen: Dieser Teil enthält Instruktionen für den Untersucher und ebenfalls Instruktionen, die dem Patienten vorgelesen werden müssen. Er verbindet patientenbezogene Informationen mit klinischen Beobachtungen und Einschätzungen des Untersuchers. Teil IV wird durch den Untersucher ausgefüllt und umfasst 6 Items.
Jedes Item wird mit 0 his 4 Punkten bewertet, wobei 0 = normal, 1 = leicht, 2 = mild, 3 = moderat und 4 = schwer ist.

\subsubsection{\acl{MoCa} (\acs{MoCa})}
\acl{MoCa} is a screening tool that can quickly identify hints on cognitive decline according to mild cognitive impairment or dementia. The entire test consists of approximately 10-minutes of questions around different domains of cognitiion. The 30 questions test cognitive abilities such as memory, language production , contextual thinking, attention and concentration, behavior, arithmetic, temporal and spatial orientation, and the ability to recognize complex shapes and patterns. The test is validated is extensively applied in clinical routine. 
% TODO add details from https://www.sralab.org/rehabilitation-measures/montreal-cognitive-assessment

\subsubsection{\acl{NMSQ}}
Die NMSS ist eine aus 30 Punkten bestehende Rater-basierte Skala zur Beurteilung eines breiten Spektrums nicht-motorischer Symptome bei Patienten mit Parkinson-Krankheit (PD). Der NMSS misst die Schwere und Häufigkeit nicht-motorischer Symptome in neun Dimensionen.

\subsubsection{\acl{BDI} (\acs{BDI})}
The \ac{BDI} is a questionnaire which aims at assessing the severity of depressive symptoms in case depression exists. It is not intended to assess depression per se, but only its severity. Hence, it cannot be used as a screening method in the normal population so taht other alternatives should be contemplated. The applied second version of the \ac{BDI} consists of  21 questions which are supposed to be evaluated for the previous two weeks. 

Scores:
\begin{itemize}
\item 0–12: no depressive symptoms or clinically inapparent 
\item 13–19: mild depressive syndrome
\item 20–28: moderate depressive syndrome
\item $>$ 29 severe depressive syndrome
\end{itemize}

\subsubsection{\acl{CBI}}
\subsubsection{\acl{CISS}}
\subsubsection{\acl{MFI-20}}
\subsubsection{\acl{PDCB}}
\subsubsection{\acl{PHQ}}
\subsubsection{\acl{PSS}}
\subsubsection{\acl{WHOQoL}}
\subsubsection{\acl{ZBI-22}}



\subsection{Baseline visit \ac{PD}-patients}
All potential candidates will undergo screening procedures as listed in Section \ref{subsec:screening} to determine their eligibility in the study. Subjects may neither have to be on stable anti-parkinsonian medications prior to informed consent nor have to be regularly treated at the \UKGM. Those subjects who meet all inclusion criteria and none of the exclusion criteria (cf. \ref{sec:study_selection}) may be enrolled. The baseline visit may occur anytime within the screening period and will serve as the final  determination of eligibility in the study. 

The following data from questionnaires should be collected from patients:
\begin{itemize}
\item General Assessments
\begin{itemize}
\item Demographic data and personal information
\item Medication schedule
\end{itemize}
\item \ac{CHAPO-PD}
\item \ac{NMSQ}
\end{itemize}


\subsection{Half year visit \ac{PD}-patients ($\pm$ 100 days)}
% Wirklich 100 Tage oder wie sind die Daten definiert?

\subsection{Annual visit \ac{PD}-patients ($\pm$ 100 days)}
% Wirklich 100 Tage oder wie sind die Daten definiert?

\subsection{Baseline visit relatives}
This study is intended as inclusion of diades of patients and relativesAll potential candidates will undergo screening procedures as listed in Section \ref{subsec:screening} to determine their eligibility in the study. Subjects may neither have to be on stable anti-parkinsonian medications prior to informed consent nor have to be regularly treated at the \UKGM. Those subjects who meet all inclusion criteria and none of the exclusion criteria (cf. \ref{sec:study_selection}) may be enrolled. The baseline visit may occur anytime within the screening period and will serve as the final  determination of eligibility in the study. 

For the relatives, the following data from questionnaires should be collected:
\begin{itemize}
\item General Assessments
\begin{itemize}
\item Demographic data and personal information
\item Relationship to patients
\item Experiencing respect in the patient-family relationship
\end{itemize}
\item \acl{BDI}, (part II)
\item \ac{CBI}
\item \ac{CISS}
\item \ac{MFI-20}
\item \ac{MoCa}
\item \ac{PDCB}
\item \ac{PHQ}
\item \ac{PSS}
\item \ac{WHOQoL}
\item \ac{ZBI-22}
\end{itemize}

\subsection{Half year visit relatives ($\pm$ 100 days)}

\subsection{Annual visit relatives ($\pm$ 100 days)}
% Welchen ``Spielraum'' ermöglichen wir den Patient:innen für die nächste Messung?

\subsection{\ac{MRI}}
Every \ac{PD}-patient will receive MR-imaging if no contraindication exists and at the request of the respective patient. With the aim of producing the greatest possible synergistic effects with other large studies at the centre and to ensure a high quality of the sequences, the programme to be run was based on the PPMI study (\url{https://www.ppmi-info.org/}). Further details are disclosed below.

\subsubsection{Overview of MR-imaging}
\begin{table}[h]
\caption{Overview on the \ac{MRI}-sequences in use during the \textsc{HessenKohorte}*}
\begin{tabularx}{1\textwidth}{@{}X *{1}{C}@{}}
\toprule
\textbf{Sequence Name} 						& \textbf{Series Description }	\\
\midrule
T1-weighted, 3D volumetric sequence 			& 3D T1-weighted 		\\
2D Gradient-echo T2*-weighted EPI (BOLD) 		& rsfMRI\_RL 			\\
Repeat 2D Gradient-echo T2*-weighted EPI (BOLD) 	& rsfMRI\_LR 			\\
NM-MT 										& 2D GRE-MT 			\\
DTI 											& DTI\_RL 			\\
Repeat DTI 									& DTI\_LR 			\\
3D T2 FLAIR 									& 3D T2 FLAIR 		\\
\bottomrule
\multicolumn{2}{l}{\footnotesize{*protocol is identical to the one used by the \ac{PPMI}-study}}
\end{tabularx}
\end{table}

\subsubsection{Procedure of the imaging}
% Werden NUR die Patienten gemessen oder auch die Angehörigen?
Participants should be positioned comfortably and correctly to minimize motion during the scan. Furthermore, technicians will be instructed to comply with the following:
\begin{itemize}
\item Participant should be informed about the total acquisition time and positioned for maximum comfort.
\item Subjects must be positioned comfortably and supine in the head coil to minimize any motion during the scan.
\item Proper back support, and support under the knees will ensure greater comfort, and lead to less motion in the scan.
\item There should be no left-right or ear-to-shoulder head tilt, and the participant’s neck should not be hyper- extended or retracted.
\item Subject's head should be centered in the head coil using the nasion (see example to the right) as an anatomical landmark. Ensure the participant is high enough in the coil to avoid loss of signal at the inferior aspects of the brain.
\item Immobilization devices, such as velcro straps, or foam padding should be used to reduce motion.
\item The positioning lasers should be used to send the nasion to the magnets isocenter.
\end{itemize}
If a participant’s neck length is such that it does not permit proper positioning in the head coil, please document this on the \ac{MRI} Acquisition Document along with any other pertinent information regarding the participants scanning session.
% Brauchen wir ein Acquisition Document?

\subsubsection{T1-weighted, 3D volumetric sequence}
\begin{table}[H]
\caption{Details on T1-weighted \ac{MRI}-sequence}
\begin{tabularx}{1\textwidth}{@{}X *{1}{X}@{}}
\toprule
\multicolumn{2}{l}{\textbf{T1-weighted, 3D volumetric \ac{MRI}-sequence during the \textsc{HessenKohorte}, e.g. \ac{MP-RAGE}, \ac{IR-FSPGR}}} \\
\midrule
Series description 								& 3D T1-weighted 											\\
Plane	 									& Sagittal 												\\
Slice thickness (mm) 							& 1.0 (slice thickness must remain consistent across timepoints) 	\\
Number of slices 								& 192 (slice thicksness may be adjusted to 1.2 mm to cover brain iff absolutely necessary. No adjustments of number of slices) 			\\
Voxel size (mm) 								& 1.0*1.0 mm in plane resolution \\
Phase encode direction 						& Anterior Posterior (AP) 			\\
Matrix										& 256 $\times$ 256 (the use of interpolation, zero-filling or a ZIP factor is not permitted)\\
TR/TE/FA/ other parameters 					& Will be defined by Invicro according to the scanner\\
FoV		 									& 256 mm (full FoV required, no rectangular FoV)\\
Scan time 									& $\sim$ 7 min\\
Further explanations 							& The FOV must include the entire brain anatomy including the vertex, cerebellum and pons. Slices should be oblique sagittal, angled along the longitudinal fissure on both the axial and coronal localizers. To avoid artifacts, position the participant such that there is sufficient empty space around the head: approximately 1.5 cm of air or more above the top of the head, and leave 3 - 4 blank slices on either side of the head. Avoid nose ghosting.\\
\bottomrule
\multicolumn{2}{l}{\footnotesize{*protocol is identical to the one used by the \ac{PPMI}-study}}
\end{tabularx}
\end{table}

\subsubsection{2D Gradient-echo T2*-weighted EPI}
\begin{table}[H]
\caption{Details on T2-weighted \ac{MRI}-sequence}
\begin{tabularx}{1\textwidth}{@{}X *{1}{X}@{}}
\toprule
\multicolumn{2}{l}{\textbf{2D-Gradient-echo T2*-weighted \ac{EPI} (e.g., ep2d\_BOLD)}} \\
\midrule                                                                                                                                                                                                                                                                                                                                                                                                                                                                                                                                                                                                                                                                                                                          
Series Description                                				& rsfMRI\_RL \\
Plane                                             					& Axial Oblique, plane parallel to AC-PC line \\
Slice thickness (mm)                              				& 3.5 with no gap \\
Number of Slices                                  				& $\sim$40 \\
Phase encode dir.                                 				& R \textgreater{}\textgreater L\\
Matrix                                            					& 64 $\times$ 64 \\
FOV                                               						& 224 $\times$ 224 mm \\
Repetition Time (ms)                              				& 2500 \\
Echo Time (ms)                                    				& 30 \\
Flip angle                                        					& 80 \\
Slice order                                       					& Interleaved \\
Number of measurements                            			& 240 (10 min total scan time) \\
In-plane acceleration                             				& GRAPPA or SENSE (factor of 2) \\
Instructions                                      					& Keep the eyes open and remain still \\
Scan Time                                         					& $\sim$10 min \\
Further explanations                              				& Please instruct the participant to keep their eyes open during the entire scan. You can instruct them to focus on a point on the mirror or scanner. Check with the participant immediately after the scan to verify they kept their eyes open and did not fall asleep. No audio or video presentation should be made during the scan.Position the axial resting state fMRI slices along the AC-PC plane with care that there is one slice above the vertex, and then cover the rest of the brain and as much of the cerebellum as possible with the remaining slices. The slices should be centered in the axial plane to prevent aliasing in the Anterior/Posterior direction (see Figure 4 ??). \ac{TR}/\ac{TE} should not be changed. \\
\bottomrule
\end{tabularx}
\end{table}

\subsubsection{REPEAT 2D Gradient-echo T2*-weighted EPI}
\begin{table}[H]
\caption{Details on REPEAT T2-weighted \ac{MRI}-sequence}
\begin{tabularx}{1\textwidth}{@{}X *{1}{X}@{}}
\toprule
\multicolumn{2}{l}{\textbf{REPEAT 2D Gradient-echo T2*-weighted \ac{EPI}}} \\
\midrule                                                                                                                                                                                                                                                                                                                                                                                                                                                                                                                                                                                                                                                                                                                          
Series Description                                                                	& rsfMRI\_LR                                  \\
Plane                                                                                      	& Axial Oblique, plane parallel to AC-PC line \\
Slice thickness (mm)                                                          	& 3.5 with no gap                             \\
Number of Slices                                                      		& $\sim$40                                    \\
Phase encode dir.                                                                 	& L \textgreater{}\textgreater R              \\
Matrix                                                                                     	& 64x64                                       \\
FOV                                                                                        	& 224 x 224 mm                                \\
Repetition Time (ms)                                                             & 2500                                        \\
Echo Time (ms)                                                                        & 30                                          \\
Flip angle                                                                                 	& 80                                          \\
Slice order 									& Interleaved                                 \\
Number of measurements                                                  & 10 (25 sec total scan time)                 \\
In-plane acceleration                                                             & GRAPPA or SENSE (factor of 2)               \\
Instructions									& Keep the eyes open and remain still         \\
Further explanations                                                             & Repeat the above scan with the phase encoding direction updated to L >> R, and the number of measurements updated to “10”. All other parameters should be held constant. Recommended imaging parameters for the repeat resting state fMRI sequence can be referenced in Table 6.                                            \\
\bottomrule
\end{tabularx}
\end{table}

\subsubsection{2D Gradient recalled echo with MT preparation}

\begin{table}[H]
\caption{Details on REPEAT T2-weighted \ac{MRI}-sequence}
\begin{tabularx}{1\textwidth}{@{}X *{1}{X}@{}}
\toprule
\multicolumn{2}{l}{\textbf{2D Gradient-echo T2*-weighted \ac{EPI} (eg ep2d\_BOLD)}} \\
\midrule                                                                                                                                                                                                                                                                                                                                                                                                                                                                                                                                                                                                                                                                                                                          
Series Description                                				& rsfMRI\_RL                                  \\
Plane                                             					& Axial Oblique, plane parallel to AC-PC line \\
Slice thickness (mm)                              				& 3.5 with no gap                             \\
Number of Slices                                  				& $\sim$40                                    \\
Phase encode dir.                                 				& R \textgreater{}\textgreater L              \\
Matrix                                            					& 64x64                                       \\
FOV                                               						& 224 x 224 mm                                \\
Repetition Time (ms)                              				& 2500                                        \\
Echo Time (ms)                                    				& 30                                          \\
Flip angle                                        					& 80                                          \\
Slice order                                       					& Interleaved                                 \\
Number of measurements                            			& 240 (10 min total scan time)                \\
In-plane acceleration                             				& GRAPPA or SENSE (factor of 2)               \\
Instructions                                      					& Keep the eyes open and remain still         \\
Scan Time                                         					& $\sim$10 minutes                            \\
Further explanations                              				& Please instruct the participant to keep their eyes open during the entire scan. You can instruct them to focus on a point on the mirror or scanner. Check with the participant immediately after the scan to verify they kept their eyes open and did not fall asleep. No audio or video presentation should be made during the scan.                                           
\end{tabularx}
\end{table}


\subsubsection{2D Diffusion-weighted EPI}
\begin{table}[H]
\caption{Details on 2D Diffusion-weighted EPI}
\begin{tabularx}{1\textwidth}{@{}X *{1}{X}@{}}
\toprule
\multicolumn{2}{l}{\textbf{2D Gradient-echo T2*-weighted \ac{EPI} (eg ep2d\_BOLD)}} \\
\midrule                                                                                                                                                                                                                                                                                                                                                                                                                                                                                                                                                                                                                                                                                                                          
2D Diffusion-weighted EPI &                                                                                      \\
Series Description        							& \ac{DTI}\_RL (and DTI\_LR for the repeated scan with reverse PE)                          \\
Plane                    						 		& Straight Axial                                                                       \\
Slice thickness (mm)      							& 2.0 with no gap                                                                      \\
Number of Slices          							& $\sim$80                                                                             \\
Phase encode dir.         							& R \textgreater{}\textgreater L                                                       \\
Matrix                    								& 128x128*                                                                             \\
FOV                       								& 256x256 mm                                                                           \\
Repetition Time (ms)      						& $\sim$10000                                                                          \\
Echo Time (ms)            							& $\sim$80                                                                             \\
Flip angle                								& 90                                                                                   \\
Slice order               								& Interleaved                                                                          \\
Number of directions      						& 32                                                                                   \\
b-VALUE                   								& 0 and 1000 s/mm2 (B=0 images interleaved throughout if possible in product sequence) \\
Instructions              							& Keep still                                                                           \\
Scan Time                 								& $\sim$8 minutes                                                                      \\
Further explanations      						& Please instruct the participant to keep still during the entire scan. \ac{DTI} should be acquired with 32 directions. Slices should cover top of the brain down to base of cerebellum. Two sequences with reversed phase encoding direction should be acquired in full to correct for susceptibility induced distortions. If acquiring a phantom scan, only one sequence with reverse phase encoding direction should be acquired                                                                                     
\end{tabularx}
\end{table}

\subsubsection{3D T2 \ac{FLAIR} Sequence}
\begin{table}[H]
\caption{Details on T2-weighted \ac{FLAIR} Sequence}
\begin{tabularx}{1\textwidth}{@{}X *{1}{X}@{}}
\toprule
\multicolumn{2}{l}{\textbf{3D T2 \ac{FLAIR} Sequence}} \\
\midrule                                                                                                                                                                                                                                                                                                                                                                                                                                                                                                                                                                                                                                                                                                                          
Series Description        							& 3D T2 FLAIR                                                                               \\
Plane                   	 		 					& Sagittal                                                                                  \\
Slice thickness (mm)      							& 1.0 – 1.2 (slice thickness must remain consistent)                                        \\
Number of slices          							& 192 (please adjust slice thickness up to 1.2 mm to cover brain, not the number of slices) \\
Voxel size (mm)           							& 1.0*1.0 mm in plane resolution                                                            \\
Phase encode dir.         							& Anterior-Posterior (AP)                                                                   \\
Matrix                    								& 256 x 256 (the use of interpolation, zero-filling or a ZIP factor is not permitted)       \\
TR/TE/FA/other parameters 					& Will be defined by Invicro according to the scanner                                       \\
FOV                       								& 256 mm (full FOV required, no rectangular FOV)                                            \\
Scan Time                 								& $\sim$7 minutes                                                                           \\
Further explanations      						&  The FOV must include the entire brain anatomy including the vertex, cerebellum and pons. To avoid artifacts, position the participant such that there is sufficient empty space around the head: approximately 1.5 cm of air or more above the top of the head, and leave 1 - 2 blank slices on top of the head. Avoid nose ghosting.                                                                                        
\end{tabularx}
\end{table}

\subsection{Biosamples}
\subsubsection{Hair}
\subsubsection{Saliva}
\subsubsection{Urine}
\subsubsection{Blood}
\subsubsection{Stool}
% SOP für die Untersuchungen durch SJ/EM

\section{Statistical considerations}

\section{Data management}

\section{Amendments}
In case of protocol changes possibly affecting the rights, safety or welfare of any subjects or scientific integrity of the data, a protocol amendment will be completed. Appropriate approvals (especially from the \ac{EC}) of the revised protocol must be obtained prior to its implementation.

\section{Compliance}
\subsection{Statement of Compliance}
This study will be conducted in accordance with ICH-GCP and with the ethical principles originating in the Declaration of Helsinki. 

\subsection{Investigator responsibilities}

\subsubsection{Delegation of responsibilities}
When specific tasks are delegated, the Principal Investigator is responsible for providing appropriate training if necessary and adequate supervision of those to whom tasks are delegated. The investigator is accountable for regulatory violations resulting from failure to adequately supervise the conduct of the clinical study. 

\subsection{Ethics committee}
The investigational site has obtained the approval of the local ethics commitee for the clinical investigation. A copy of the written approval of the protocol can be found in the Appendix (cf. chapter \ref{}). Any amendment to the protocol will require review and approval by the ethics commitee before any changes are implemented to the study. Besides, all changes to the \ac{ICF} will have to be approved, as well. In case of an extension of the study to further centers, an ethics approval must be obtained by the respective ethics commitee. 

\section{Monitoring}
% Wie gehen wir mit AE/SAE um? Braucht man das überhaupt?

\section{Potential Risks and Benefits}

\subsection{Anticipated Adverse Events}

\subsection{Risks associated with the study participation}

\subsection{Risks associated with the \ac{MRI}}

\section{Safety Reporting}

\section{Informed consent}

\section{Suspension or termination of the study}

\section{Study registration and Results}

\section{Bibliography}

\chapter{Appendix}
% Metallanamnesebogen
% Bogen für Probandenuntersuchungen
% Aufklärung als pdf einfügen
% Master Log
% Screening Log
% Subject Visit Log
% Informed Consent Log
% Biobank Telephone contact form
% Third parties telefone contact form
\end{document}